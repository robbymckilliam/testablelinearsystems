We have so far studied linear shift-invariant systems and inparticular those systems described by linear differential equations with constant coefficients.  Such systems are useful for modelling electrical circuits, mechanical machines, electro-mechanical devices, and many other real world devices.  %One particular linear time invariant system has so far been absent. 
The time-shifter $T_\tau$ with non zero time shift $\tau \neq 0$ has so far been absent.  %These systems describe pure delays or advances in time of a signal.  %It turns out to be difficult to construct systems, for example, an electrical circuit, that accurately model a time shifter $T_\tau, \tau \neq 0$.

We now consider linear shift-invariant systems constructed from linear combinations of time-shifters of the form $T_{Pm}$ where $m \in \ints$ and $P$ is a positive real number called the~\term{sample period} or simply the \term{period}.  That is, we consider systems of the form
\begin{equation}\label{eq:discretimesysdefn}
H x = \sum_{m \in \ints} h_m T_{ P m} x
\end{equation}
where $h \in \ints \to \complex$ is complex valued sequence.  Such systems are called \term{discrete time systems}.  It is common in the literature that the term discrete time system refers to a mapping from a complex valued sequence $x$ to another complex valued sequence $y$ of the form
\[
y_n = \sum_{m \in \ints} h_m x_{n-m}.
\]
We can recover this definition from~\eqref{eq:discretimesysdefn} by setting the period $P = 1$ and identifying $y_n$ with $Hx(n)$ and $x_n$ with $x(n)$.  In this way,~\eqref{eq:discretimesysdefn} is a more general definition of discrete time system.  We will find that~\eqref{eq:discretimesysdefn} provides a setting for the study of discrete time systems that is closely linked with previous chapters where what are often called \emph{continuous-time} systems have been studied.  Complex exponetial signals $e^{st}$ remain the eigenfunctions and the notions of the transfer function and spectrum carry through unchanged.  This more general formulation of discrete-time system~\eqref{eq:discretimesysdefn} is connected with what \citet[Sec.~9.5]{Zemanian_dist_theory_1965} calls the continuous variable case of a linear \term{difference equation} with constant coefficients.  We will study difference equations further in Section~\ref{sec:difference-equations}.

Discrete time systems are not regular because the time-shifter is not regular.  However, we will find that the sequence $h$ plays a role analogous to that of the impulse response of a regular system.  For this reason $h$ is called the~\term{discrete impulse response} of $H$.


\section{The discrete impulse response} \label{sec:discr-time-impulse}

We first describe a suitable domain for each discrete time system.  It is worth considering an example.  Suppose that $H$ has discrete impulse response given by the step sequence $u$~\eqref{eq:stepsequence}.  The signal $x(t) = 1$ that takes the value $1$ for all $t \in \reals$ will not be in the domain of $H$ because, in this case,
\[
Hx(t) =  \sum_{m \in \ints} u_m T_{ P m} x(t) = \sum_{m \in \ints} u_m = \sum_{m=0}^\infty 1
\]
is not finite for any $t \in \reals$.  Given a sequence $h$ denote by $\dom_P h$ the set of signals $x$ such that
\[
\sum_{m \in \ints} \abs{h_m x(t - Pm)} < \infty \qquad \text{for all $t \in \reals$}.
\]
It can be confirmed that $\dom_P h$ is a linear shift-invariant space.  If $H$ has discrete impulse response $h$ then, for all signals $x \in \dom_P h$,
\[
\abs{Hx(t)} = \abs{\sum_{m \in \ints} h_m T_{ P m} x(t)} \leq \sum_{m \in \ints}  \abs{ h_m x(t - Pm)} < \infty \qquad \text{for all $t \in \reals$}
\]
and so $Hx(t)$ is finite for all $t \in \reals$.  From here on, we take $\dom_P h$ as the domain of the discrete time system with discrete impulse response $h$ unless otherwise stated.  %Be aware that set $\dom_P h$ is now defined for both signals $h$ and sequences $h$.  %This reuse of notation is often called \term{overloading} in reference to the same concept used in many programming languages.

The discrete impulse response $h$ immediately yields some properties of the corresponding discrete time system $H$.  For example, if $h_m = 0$ for all $m < 0$, then $H$ is causal because the response 
\[
Hx(t) = \sum_{m \in \ints} h_m T_{ P m}x(t) = \sum_{m =0}^\infty h_m x(t - Pm)
\] 
at time $t$ only depends on values of the input signal $x$ at times less than or equal to $t$.  A discrete time system is stable if and only if its discrete impulse response is absolutely summable (Exercise~\ref{excer:stableimpulserespdiscretetime}).  This is analagous to the property of regular systems that are stable if and only if their impulse response is absolutely integrable (Exercise~\ref{excer:bibostableimpulseresp}).

Let $F$ and $G$ be discrete time systems with equal sample period $P$ and discrete impulse responses $f$ and $g$.  Let $a,b\in\complex$ and let $H = a F + b G$ be a system formed by linear combination of $F$ and $G$.  The response of $H$ to input signal $x \in \dom_P f \cap \dom_P g$ is
\begin{align*}
Hx &= a \sum_{n \in \ints} f_{n} T_{ P n}x + b \sum_{n \in \ints} g_{n} T_{ P n}x \\
&=  \sum_{n \in \ints} (a f_{n} + b g_{n}) T_{ P n}x,
\end{align*}
and so $H$ is a discrete time system with discrete impulse response given by the linear combination of sequences $af + ag$.  We take the domain of $H$ to be $\dom_P f \cap \dom_P g$ unless otherwise stated.

Now suppose that $Hx = F Gx$ is formed by the composition of discrete time systems $F$ and $G$.  Denote by $\dom_P f\,g$ the set of signals $x$ such that
\[
\sum_{m \in \ints} \sum_{k \in \ints} \abs{f_m g_k x(t - m - k)} < \infty \qquad \text{for all $t \in \reals$}. 
\]  
We will find $\dom_P f\,g$ to be a convenient domain for the system $H = FG$.  The response of $H$ to $x \in \dom_P f\,g$ is
\begin{align*}
H x &= \sum_{m \in \ints} f_m T_{Pm} G x  \\
&= \sum_{m \in \ints} f_m  G T_{Pm} x  \qquad \text{(shift-invariance of $G$)} \\
&= \sum_{m \in \ints} \sum_{k \in \ints} f_m g_k T_{P(m+k)}x.
\end{align*}
Because $x \in \dom_P f\,g$, Fubini's theorem~\cite[Theorem~8.8]{Rudin_real_and_complex_analysis} justifies swapping the order of summation so that
\[
H x = \sum_{k \in \ints} \sum_{m \in \ints} f_m g_k T_{P(m+k)}x
\]
and by putting $n = m+k$ we have
\begin{align}
Hx =  \sum_{n \in \ints} \sum_{m \in \ints} f_m g_{n-m} T_{Pn}x = \sum_{n \in \ints} h_n T_{Pn}x \nonumber
\end{align}
where $h$ is the sequence with elements given by
\[
h_n = \sum_{m \in \ints} f_m g_{n-m} = (f * g)_n .
\]
This is called the \term{discrete convolution} of sequences $f$ and $g$.  The notation $f * g$ is reused to denote discrete convolution so that $h = f * g$.  We have found that the system $H$ constructed by composition of the discrete time systems $F$ and $G$ is a discrete time system.  The discrete impulse response of $H$ is the discrete convolution of the discrete impulse responses of $F$ and $G$.  The domain of $H$ is taken to be $\dom_P f\,g$ unless otherwise stated.  

Discrete convolution has properties analogous to that of the convolution of signals described in Section~\ref{sec:prop-conv}.  For example, discrete convolution is commutative and associative under approriate assumptions (Exercise~\ref{excer:discrconvassociative}).
% \[
% (f*g)*h = f*(g*h). \qquad \text{(Exercise~\ref{excer:discrconvassociative})}
% \] 
% Discrete convolution distributes with addition and commutes with scalar multiplication, that is,
% \[
% a (f * h) + b (g * h) = (af + bg) * h
% \]
% where $a$ and $b$ are real or complex constants.  
In what follows we will often use the term convolution and impulse response rather than the lengthier terms discrete convolution and discrete impulse response whenever there is no chance for confusion.

% Consider the convolution $u * u$ of the step sequence~\eqref{eq:stepsequence} with itself.  The $n$th element of the convolution is
% \[
% (u * u)_n = \sum_{m \in \ints} u_m u_{n-m} = \sum_{m = 0}^n 1 = \begin{cases}
% n+1 & n \geq 0 \\
% 0 & n < 0.
% \end{cases}
% \] 
% Not all sequences can be convolved.  Denote by $\onebf$ the sequence with all elements equal one.  The convolution $u * \onebf$ is not possible because
% \[
% (u * \onebf)_n =  \sum_{m \in \ints} u_m \onebf_{n-m} = \sum_{m = 0}^\infty 1 = \infty
% \]
% is not finite for any $n$.  The convolution $u * \onebf$ is said not to exist.  When considering convolution of sequences it is important to make sufficient assumptions for the convolution to exist.  For example, the the convoltuion $f * g$ always exists if both $f$ and $g$ are absolutely summable sequences (Exercise~\ref{exer:convabssummableisabssummable}).  Similarly, the interchange of summation in~\eqref{eq:discconvinterchangesummation} only holds under appropriate assumptions about the sequences $f$ and $g$.  For example, the interchange is valid when $f$ and $g$ are absolutely summable.

% Discrete convolution has many properties analogous to that of the convolution of signals described in Section~\ref{sec:prop-conv}.  For example, discrete convolution is commutative, that is, $f * g = g * f$.  Discrete convolution is associative, that is, if $f$, $g$, and $h$ are sequences then
% \[
% (f*g)*h = f*(g*h). \qquad \text{(Exercise~\ref{excer:discrconvassociative})}
% \] 
% Discrete convolution distributes with addition and commutes with scalar multiplication, that is,
% \[
% a (f * h) + b (g * h) = (af + bg) * h
% \]
% where $a$ and $b$ are real or complex constants.  

% Discrete time system have particularly convenient properties when operating upon bandlimited signals.  Let $H$ be a discrete time system with impulse response $h$ and sample period $P$ and let $x$ be a bandlimited signal with bandwidth $b \leq \tfrac{F}{2} = \tfrac{1}{2P}$.  Let $c_n = x(nP)$ be the sequence of samples of $x$ so that
% \[
% x(t) = \sum_{n \in \ints} c_n \sinc(Ft - n).
% \]
% The response of $H$ to $x$ is
% \begin{align*}
% y = H(x) &= \sum_{n \in \ints} h_n T_{Pn}\left(\sum_{m \in \ints} c_m \sinc(Ft - m) \right) \\
% &=  \sum_{m \in \ints} \sum_{n \in \ints} h_n c_m \sinc(Ft - n - m),
% \end{align*}
% the interchange of summation being valid under appropriate assumptions about the sequence s$h$ and $c$.  Putting $k = m+n$ we have
% \[
% y = H(x) = \sum_{m \in \ints} \sum_{k \in \ints} h_n c_{k-n} \sinc(Ft - k) =  \sum_{m \in \ints} d_k \sinc(Ft - k)
% \]
% where $d = h * c$, that is, $d$ is the sequence with $n$th element
% \[
% d_n = (h * c)_n = \sum_{m \in \ints} h_m c_{n-m}.
% \] 
% We see that the response $y = H(x)$ is a bandlimited signal with samples $d_n = y(nP)$.


\section{The transfer function and the spectrum}

Recall that complex exponential signals are eigenfunctions of linear shift-invariant systems, that is, if $H$ is a linear shift-invariant system with domain $X$, then the response of $H$ to input signal $e^{st} \in X$ satisfies
 \[
He^{st} = \lambda H(s) e^{st}
\]
where $\lambda H(s)$ is a complex number that depends on $s \in \complex$, but not on $t \in \reals$.  Considered as a function of $s$, the expression $\lambda H$ is called the transfer function of the system $H$.  The domain of $\lambda H$ is the set of complex numbers $s$ such that the signal $e^{st} \in X$ and this set is denoted by $\cep X$.

Let $H$ be a discrete time system with discrete impulse response $h$ and domain $\dom_P h$.  The response of $H$ to $e^{st} \in \dom_P h$ satisfies
\begin{align*}
H e^{st} &= \sum_{n \in \ints} h_n T_{Pn} e^{st} \\
&= \sum_{n \in \ints} h_n e^{s(t-Pn)} \\
&= e^{st} \sum_{n \in \ints} h_n e^{-sPn} = e^{st} \lambda H(s).
\end{align*}
It follows that the transfer function of $H$ satisfies
\[
\lambda H(s) = \sum_{n \in \ints} h_n e^{-sPn} \qquad s \in \cep \dom_P h.
\]

For example, consider the discrete time system $H$ with discrete impulse response given by the step sequence $u$.  The domain of the transfer function $\cep \dom_P u$ contains all those complex numbers $s$ for which
\[
\sum_{n \in \ints} \abs{u_n e^{-sPn}} = \sum_{n = 0}^\infty e^{-\Re s Pn} < \infty.
\]
This is precisely those $s \in \complex$ with positive real part.  The transfer function is
\[
\lambda H(s) = \sum_{n \in \ints} u_n e^{-sPn} = \sum_{n = 0}^\infty e^{-sPn} = \frac{1}{1 - e^{-sP}} \qquad \Re s > 0.
\]
The identity system $T_0$ is a discrete time system with discrete impulse response given by the delta sequence
\[
\delta_n = \begin{cases}
1 & n = 0 \\
0 & \text{otherwise}.
\end{cases}
\]  
The transfer function is 
\[
\lambda T_0(s) = \sum_{n \in \ints} \delta_n e^{-sPn} = 1 \qquad s \in \complex.
\]
The domain of $\lambda T_0$ is the entire complex plane $\complex$.  This is in agreement with \eqref{eq:timeshiftertransferfunction}.  % Finally consider the discrete time system $H$ with impulse response 
% \[
% h_n = \rect(\tfrac{n}{2W}) \begin{cases}
% 1 & \abs{n} < W \\
% 0 & \text{otherwise}
% \end{cases}
% \]
% where $W \in \reals$.  The transfer function is
% \begin{align*}
% \lambda H(s) = \sum_{n \in \ints} h_n e^{-sPn} &= \sum_{n = -W + 1}^{W-1} e^{-sPn} \\
% &= \sum_{k = 1}^{2W-1} e^{-sPn} \qquad \text{(change variable $k = n + W$)} \\ 
% &= \frac{e^{-2WsP} - e^{-sP}}{e^{-sP} - 1} \qquad \text{Exercise~\ref{excer:sumgeomeebeta}}
% \end{align*}
% for all $s \in \complex$.  The domain is again the entire complex plane.

%TODO: examples.

Now suppose that $H$ is a stable discrete time system with absolutely summable discrete impulse response $h$.  The oscillatory complex exponential signal $e^{2\pi j f t}$ is in the domain $\dom_P h$ for all $f \in \reals$ because
\[
\sum_{m \in \ints} \abs{h_m e^{2\pi j f (t - m)}} = \sum_{m \in \ints} \abs{h_m} < \infty \qquad \text{for all $t \in \reals$}.
\]
Equivalently, the domain of the transfer function $\lambda H$ contains the imaginary axis, that is, the set $\cep \dom_P h$ contains the imaginary axis.  The spectrum of $H$ is
\[
\Lambda H(f) = \lambda H (j2\pi f) = \sum_{n \in \ints} h_n e^{-2\pi j f P n} \qquad f \in \reals.
\]
Observe that the spectrum $\Lambda H$ is a periodic signal with period equal to the reciprocal of the sample period $F = \frac{1}{P}$ called the \term{sample rate}.  

The spectrum is related to the discrete time Fourier transform~\eqref{eq:disctransintro} of $h$ by
\[
\calD h(f) = \Lambda H(\tfrac{f}{P}) =\sum_{n \in \ints} h_n e^{-2\pi j f n}.
\]
%TODO. \calD h isa defined even when h is square integrable and not necessary abs integrable, i.e., sinc functions etc.  Some example where the order summation is taken is important would be a good idea. 
%TODO: Examples, rect, sinc, truncated sinc, truncated rrc. 
As an example, suppose that $H$ is the system with period $P$ and discrete impulse response
\[
h_n = \rect(n/2) = \begin{cases}
1 & \abs{n} \leq 1 \\
0 & \text{otherwise}.
\end{cases}
\]
The spectrum of $H$ is
\[
\Lambda H(f) = \calD h(Pf) = e^{-2\pi j P f} + 1 + e^{2\pi j P f} = 1 + 2\cos(2\pi P f).
\]
The sequence $\rect(n/2)$ and its discrete Fourier transform are plotted in Figure~\ref{fig:discretetimefouriertransexamples}.
% Now suppose that $H$ has discrete impulse response $h_n = \sinc(n/4)$ given by sampling the sinc function with period $\tfrac{1}{4}$.  The Fourier transform of $\sinc(t/4)$ is $4 \calF\sinc(4f) = 4\rect(4f)$ by the time scaling property~\eqref{eq:timescalingpropertryfouriertrans}.  The discrete time Fourier transform of $h$ is given by periodically summing this Fourier transform~\eqref{eq:dftsignalperiodicsumP}, that is,
% \[
% \calD h(f) = 4 \sum_{m\in\ints} \rect(4f - m).
% \]
% We are lead to believe that the spectrum of this system $H$ is
% \[
% \lambda H(f) = \calD h(Pf) = 4 \sum_{m\in\ints} \rect(4Pf - m).
% \]
% %This is plotted in Figure~\ref{}.  
% 

\section{Finite impulse response filters}\label{sec:finite-impulse-resp}

We have found that the spectrum of a stable discrete time system is periodic with period equal to the sample rate $F = \tfrac{1}{P}$.  Because of this periodicity it is not immediately apparent that any useful filters can be constructed from discrete time systems.  For example, periodicity makes it impossible to build a lowpass filter.  However, many signals occuring in practice are approximately bandlimited with some bandwidth $b > 0$ (Section~\ref{sec:bandlimited-signals}).  We will find that it is possible to perform useful filtering operations on bandlimited signals using discrete time systems.  This is most easily achieved when the bandwidth is less than half the sample rate of the system, that is, when $b < \tfrac{F}{2}$.

To see how this works, consider the bandlimited signal $x$ with Fourier transform
\[
\hat{x}(f) = \calF x(f) = \rect(f) - \cos(2\pi f) \big( \rect(2f - 3) + \tfrac{1}{2} \rect(2f + 3) \big)
\]
that is plotted by the dashed line in Figure~\ref{fig:bandlimiteddiscretetimefiltering}.  The Fourier transform has a rectangular pulse centered at the origin and raised cosine pulses of width $\tfrac{1}{2}$ at $f = \pm \tfrac{3}{2}$.  The signal is bandlimited with bandwidth $\tfrac{7}{4}$.  Suppose that the component of interest is the rectangular pulse at the origin and that the two raised cosine pulses would preferably not exist.  A low pass filter could be used to filter out the rectangular pulses.  For example, an ideal lowpass filter (Section~\ref{sec:analogue-filters}) with cuttoff frequency $c$ in the interval $(\tfrac{1}{2}, \tfrac{5}{4})$ and spectrum $\rect(\tfrac{f}{2c})$ would preserve the recangular pulse at the origin and remove the two raised cosine pulses.  The impulse response of this ideal lowpass filter is $2c\sin(2ct)$.

Because the signal has bandwidth $b = \tfrac{7}{4}$ the spectrum of the filter for frequencies $\abs{f} > \tfrac{7}{4}$ is not important.  By exploting this property will be find that the same lowpass filtering can be achieved with a discrete time system.  Let $H$ be the discrete time system with discrete impulse response $h_n = 2cP \sinc(2c Pn)$ and period $P = \tfrac{1}{F}$ with $F > 2b = \frac{7}{2}$.  The discrete time Fourier transform of $h$ is given by~\eqref{eq:dftsignalperiodicsumP},
\[
\hat{h}(f) = \calD h(f) = \sum_{m \in \ints} \rect\left( \frac{Ff + Fm}{2c}\right).
\] 
We are lead to believe that the spectrum of $H$ is
\begin{equation}\label{eq:LambdaHidealklowpassdigitalsceptical}
\Lambda H(f) = \hat{h}(Pf) = \sum_{m \in \ints} \rect\left( \frac{f + Fm}{2c}\right).
\end{equation}
The spectrum is periodic with period $F > 2b = \tfrac{7}{2}$ and is plotted by the shaded region in Figure~\ref{fig:bandlimiteddiscretetimefiltering} in the case that $c = 1$ and $F = 4$.  The system $H$ is called an \term{ideal lowpass digital filter} with cuttoff frequency $c$ and sample period $P$.  

The conclusion that the spectrum of $H$ satisfies~\eqref{eq:LambdaHidealklowpassdigitalsceptical} needs to be taken with some scepticism because the discrete impulse response $h_n = 2cP\sinc(2c P n)$ is not absolutely summable and so the domain $\cep \dom_P h$ does not contain the imaginary axis.  Like ideal analogue filters (Section~\ref{sec:analogue-filters}), ideal digital filter will not be useful in practice and so this caveat will not affect the results that follow. The concept of an ideal digital filter is nonetheless a valuable aid to intuition.  

We now turn our attention to practical digital filters.  The first of these are called \term{finite impulse response filters} or FIR filters.  One feature preventing practical implementation of ideal digital filters is that the discrete impulse response $2c\sin(2cPn)$ is not of finite support.  Computing the response $\sum_{m \in \ints} h_m T_{Pm} x(t)$ for any time $t \in \reals$ might require an infinite number of operations.  A pragramtic approach to alleviate this problem is to first approximate the ideal analogue lowpass filter by one with impulse response
\[
2c w(t) \sinc(2ct)
\]
where $w(t)$ is called a \term{window function} (sometimes also called an \term{apodisation function}) that is zero outside an interval $(-W/2, W/2)$ where $W > 0$ is the \term{window width}.  Because of the window function this impulse response has finite support.  Common window functions include the rectangular window $w(t) = \rect(t/W)$, the raised cosine window
\[
w(t)  = \frac{1}{2} \rect\left(\frac{t}{W}\right)\left( 1 + \cos\left(\frac{2\pi t}{W}\right) \right)
\]
also known as the \term{Hann window}, and the \term{Blackman window}~\cite{BlackmanTukeywindows}
\[
w(t) = \rect\left(\frac{t}{W}\right) \left( a_0 + a_1 \cos\left(\frac{2\pi t}{W}\right) + a_2 \cos\left(\frac{4\pi t}{W}\right) \right) 
\]   
where $a_0 = \tfrac{21}{50}$, $a_1 = \tfrac{1}{2}$, and $a_2 = \tfrac{2}{25}$.


\begin{test}\label{test:firfilteredlecturerec}
(\textbf{Finite impulse response filtered lecture recording})

We consider again the \SI{20}{\second} audio signal from Test~\ref{test:ftlecturerec}.

\end{test}

% A crude low pass digital filter can be constructed from the discrete time system corresponding with a first order difference equation~\eqref{eq:firstorderdifferenceequation}.  Consider the first order difference equation
% \[
% c_n = 3 d_n - 2 d_{n-1}.
% \]
% Let $H$ be the discrete time system with sample rate $F = \tfrac{1}{P} = 4$ corresponding with this difference equation.  The spectrum of $H$ is
% \[
% \Lambda(H,f) = \frac{e^{\pi f j/2}}{3 e^{\pi f j/2} - 2}.
% \]
% The magnitude spectrum is plotted in Figure~\ref{fig:bandlimiteddiscretetimefiltering} (gray).  The magnitude spectrum takes its maximum value $\abs{\Lambda(H,0)} = 1$ at $f$ a multiple of $F = 4$ and minimum value $\abs{\Lambda(H,2)} = \tfrac{1}{5}$ at frequencies $f = (k + \tfrac{1}{2})F$ for $k \in \ints$.  The magnitude spectrum takes the value $\frac{1}{\sqrt{2}}$ at frequences of the form
% \[
% f = 4k + \frac{2}{\pi} \arccos\left( \frac{13 - \sqrt{2}}{12} \right) \approx 4k + 0.167754.
% \]
% On the interval $[-F/2,F/2]$ this discrete time system acts as a lowpass filter with cuttoff frequency approximately 
% \[
% \frac{2}{\pi} \arccos\left( \frac{13 - \sqrt{2}}{12} \right) \approx 0.167754.
% \]

\begin{figure}[tp]
  \centering
  \def\sx{1.4}
  \def\sy{3}
  \def\P{0.25}
  \def\F{(1/\P)}
  \def\c{1}
  \colorlet{lightgray}{black!25}
  \begin{tikzpicture}
    \begin{scope}[yscale=\sy,xscale=\sx]
      % draw light gray region for magnitude spectrum of the discrete time system. 
      \fill[fill=lightgray] (-4.3,0) -- (-4.3,1) -- (-4+\c,1) -- (-4+\c,0) -- (-\c,0) -- (-\c,1) -- (\c,1) -- (\c,0) -- (4-\c,0) -- (4-\c,1) -- (4.3,1) -- (4.3,0) -- cycle;
      
      % draw axis
      \draw[->] (-4.3,0) -- (4.3,0) node[above] {$t$};
      % \draw[->] (0,-0.35) -- (0,2.3) node[above,font=\footnotesize] {$\rect(f) \big(1 + \cos( 2 \pi f )\big) + \rect{3f - 2} + \rect{3f + 2}$};    
      \draw[->] (0,-0.2) -- (0,1.3) node[above,font=\footnotesize] {};  
      
      % draw unfiltered Fourier transform magnitude
       \draw[thick,dashed] (-4,0) -- (-1.75,0) 
       -- plot[smooth,domain=-1.75:-1.25,samples=50] (\x,{-cos(2*pi*\x r)})
       -- (-1.25,0) -- (-0.5,0) -- (-0.5,1) -- (0.5,1) -- (0.5,0) -- (1.25, 0)
       -- plot[smooth,domain=1.25:1.75,samples=50] (\x,{-cos(2*pi*\x r)})
       -- (1.75,0) -- (4,0);
      %  
      % -- (0.5,0) -- (1.25,0) -- (1.25,1) -- (1.75,1) -- (1.75,0) -- (4,0);
      
       %draw filtered Fourier transform magnitude
       \draw[thick] (-4,0) -- (-0.5,0) -- (-0.5,1) -- (0.5,1) -- (0.5,0) -- (4,0);
      
      % add some axis ticks
    \end{scope}
    \begin{scope}[xscale=\sx]
      \vtick{-2.0} node[pos=0.5,below] {$-2$};
      \vtick{2} node[pos=0.5,below] {$2$};
      \vtick{-4.0} node[pos=0.5,below] {$-4$};
      \vtick{4} node[pos=0.5,below] {$4$};
    \end{scope}
    \begin{scope}[yscale=\sy]
      \htick{1} node[pos=0.5,above left] {$1$};
    \end{scope}
  \end{tikzpicture}
  \caption{Fourier transform $\rect(f) - \cos(2\pi f) \big( \rect(2f - 3) + \tfrac{1}{2} \rect(2f + 3) \big)$ (dashed) of a bandlimited signal with bandwidth $b < 2$. The solid line shows the magnitude of the Fourier transform of the response (solid) to this signal of the ideal low pass digital filter with cuttoff frequency $c = 1$ and sample rate $F = \tfrac{1}{P} = 4$.  The discrete impulse response of this filter is $\tfrac{1}{2} \sinc(n/2)$.  Observe that the higher frequency rasied cosine pulses centered at $f = \pm \tfrac{3}{2}$ are removed by the filter, but the rectangular pulse centered at the origin is unaffected.  The shaded region shows the periodic magnitude spectrum of the system.  The periodicity of the spectrum is not of consequence because the signal is bandlimited with bandwidth $b < \tfrac{F}{2} = 2$.} \label{fig:bandlimiteddiscretetimefiltering}
\end{figure}

\begin{itemize}

\item Periodicity of the spectrum makes it not immediately clear how discrete time systems will be useful as filter

\item However, bandlimited signals!

\item Picture of 1st order discrete time system filtering something?

\item Test with 1st order system filtering high frequency noise from lecture recording

\item better filters that 1st order are not too hard to realise

\item One approach involves mapping analogue filters, IIR filter,  Bilinear transform

\item Do butterworth as an example.  Implement high order butterworth like filter this way.

\item Filter lecture video with this IIR filter

\item Another approach has no zeros, attempt to design impulse response directly, FIR filters, windowing.

\item Filter lecture video with FIR filter.

\item IIR usually computational advantageous.  FIR better if linear phase important.

\end{itemize}


\section{The z-transform}

Recall that the transfer function $\lambda H$ of a discrete time system $H$ with impulse response $h$ satisfies
\[
\lambda H(s) = \sum_{n \in \ints} h_n e^{-sPn} \qquad s \in \cep \dom_P h.
\]
Putting $z = e^{Ps}$ we have
\begin{equation} \label{eq:reltransferfunztrans}
\lambda H(s) = \lambda H(\tfrac{1}{P} \log z) = \sum_{n \in \ints} h_n z^{-n} 
\end{equation}
that holds for complex numbers $z = e^{sP}$ such that $s \in \cep\dom_P h$.  Denote this set by $\roc_z h$, that is,
\[
\roc_z h = \{z \in \complex \mid \frac{1}{P} \log z \in \cep \dom_P h \ \}.
\]
The right hand side of~\eqref{eq:reltransferfunztrans} is called the \term{z-transform} of the sequence $h$ and the set $\roc_z h$ is called the region of convergence of the z-transform.  

%TODO include description of mapping $z = e^{sP}$ with figure.  Map real part of $s$ to magnitude and imaginary part to angle.  Origin is removed.

Let $c$ be a complex valued sequence.  The z-transform of $c$ is denoted by $\calZ c$.  It is the complex valued function of $\roc_z c$ satisfying
\[
\calZ c(z) = \sum_{n \in \ints} c_n z^{-n} \qquad z \in \roc_z c.
\]
The domain of $\calZ c$ is the region of convergence $\roc_z c$.  It happens that the region of convergence $\roc_z c$ is precisely the set of nonzero complex numbers such that the sequence $c_n z^{-n}$ is absolutely summable, that is, those $z \neq 0$ such that
\[
\sum_{n \in \ints} \abs{c_n z^{-n}} < \infty. \qquad \text{(Exercise~\ref{exer:roczabssummrel})}
\]
The z-transform of a complex valued sequence plays a role analogous to the Laplace transform of a signal.  Recall that the region of convergence of the Laplace transform was either a half plane, a vertical strip, the entire complex plane, or the empty set (Section~\ref{sec:laplace-transform}).  We will find that the region of convergence of the z-transform is either a circular disk with the origin removed, an annular region in the complex plane, the complex plane with an origin centered disc removed, the entire complex plane, or the empty set.

We now consider some example z-transforms.  Consider first the step sequence $u$.  The region of convergence $\roc_z u$ is all those complex numbers with magnitude greater than one because
\[
\sum_{n\in\ints} \abs{u_n z^{-n}} = \sum_{n=0}^\infty \abs{z}^{-n} < \infty \qquad \text{only if $\abs{z} > 1$}.
\]
Graphically this region of convergence is the complex plane with a disc of radius one centered at the origin removed (Figure~\ref{fig:rocexampleztrans1}).  The z-transform of $u$ is
\[
\calZ u(z) = \sum_{n \in \ints} u_n z^{-n} = \sum_{n=0}^\infty z^{-n} = \frac{z}{z - 1} \qquad \abs{z} > 1. \qquad \text{(Exercise~\ref{exer:stepseqZtrans})}
\]

Now consider the sequence with $n$th element $(\tfrac{1}{2})^{n} u_n$.  The region of convergence is all those complex numbers with magnitude greater than $\tfrac{1}{2}$.   The z-transform is
\[
\calZ\big( (\tfrac{1}{2})^{n} u_n\big) = \sum_{n \in \ints} (\tfrac{1}{2})^{n} u_n z^{-n} = \sum_{n=0}^\infty (2z)^{-n} = \frac{2z}{2z - 1} \qquad \abs{z} > \tfrac{1}{2} .
\]
Graphically, the region of convergence is the complex plane with a disc of radius $\tfrac{1}{2}$ removed.  Now consider the sequence with elements 
\[
(\tfrac{3}{2})^{n} u_{-n} =  \begin{cases} 
(\tfrac{3}{2})^{n} & n \leq 0 \\
0 & n > 0.
\end{cases}
\]
In this case the region of convergence is all those nonzero complex numbers with magnitude less than $\tfrac{3}{2}$ because
\[
\sum_{n\in\ints} \abs{ (\tfrac{3}{2})^{n} u_{-n} z^{-n}} = \sum_{n=0}^\infty \abs{\tfrac{2}{3}z}^{n} < \infty \qquad \text{only if $\abs{z} < \tfrac{3}{2}$}.
\]
The z-transform is
\[
\calZ\big( (\tfrac{3}{2})^n u_{-n} \big) = \sum_{n \in \ints} (\tfrac{3}{2})^{n} u_{-n} z^{-n} = \sum_{n=0}^\infty (\tfrac{2}{3} z)^{n} = \frac{3}{3 - 2z} \qquad \abs{z} < \tfrac{3}{2}
\]
The region of convergence is an open disc of radius $\frac{3}{2}$ with the origin removed.  The sequence with $n$th element $(\tfrac{1}{2})^{n} u_n + (\frac{3}{2})^{n} u_{-n}$ has z-transform
\[
\calZ\big((\tfrac{1}{2})^{n} u_n + (\tfrac{3}{2})^{n} u_{-n}\big) =  \frac{2z}{2z - 1} + \frac{3}{3 - 2z} \qquad \tfrac{1}{2} < \abs{z} < \tfrac{3}{2}
\]
with region of convergence all those complex numbers such that $\tfrac{1}{2} < \abs{z} < \tfrac{3}{2}$.  The region of convergence is an annulus (a doughnut) in the complex plane with inner radius $\tfrac{1}{2}$ and outer radius $\tfrac{3}{2}$.

The delta sequence $\delta$ has z-transform
\[
\calZ(\delta) = \sum_{n \in \ints} \delta_n z^{-n} = 1.
\]
The region of convergence is the entire complex plane with the origin removed, that is, $\roc_z \delta = \complex \\ 0$.  Finally, consider the sequence $\onebf$ that has every element equal to $1$.  Because $\sum_{n \in \ints} \abs{z}^{-n}$ does not converge for any $z \in \complex$ the region of convergence $\roc_z \onebf$ is the empty set.  The sequence $\onebf$ is said not to have a z-transform.

{
\def\minn{-4}
\def\maxn{4}
\def\ymax{4}
\def\ymin{-0.5}
\def\scalex{0.6}
\def\step(#1){(#1>=0)} %step function
\begin{figure}[tp]
\centering
\begin{tikzpicture}
  \def\scaley{3}
  \begin{scope}[xscale=\scalex,yscale=\scaley]
    %\draw[very thin,color=gray] (-0.1,-1.1) grid (3.9,3.9);
    \draw[->] (\minn-0.5,0) -- (\maxn+0.5,0) node[above] {$n$};
    \draw[->] (0,\ymin/\scaley) -- (0,\ymax/\scaley) node[left] {$u_n$};
    %\draw[color=black] plot[id=x] function{1/x^2} 
    %    node[right] {$f(t) = t^{-2}$};
    \draw[color=black,thick,ycomb,mark=*,mark options={xscale=1/\scalex,yscale=1/\scaley,scale=0.75},domain=\minn:\maxn,samples=\maxn-\minn+1] plot function{(x>=0)};
    % \draw[color=black] plot[id=exp] function{0.05*exp(x)} 
    %    node[right] {$f(t) = \frac{1}{20} e^t$};
  \end{scope}
  \htick{1*\scaley} node[pos=0.5,left] {$1$};
\end{tikzpicture} \;\;
\begin{tikzpicture}
  \def\scaley{3}
  \begin{scope}[xscale=\scalex,yscale=\scaley]
    %\draw[very thin,color=gray] (-0.1,-1.1) grid (3.9,3.9);
    \draw[->] (\minn-0.5,0) -- (\maxn+0.5,0) node[above] {$n$};
    \draw[->] (0,\ymin/\scaley) -- (0,\ymax/\scaley) node[left] {$(\tfrac{1}{2})^n u_n$};
        \draw[color=black,thick,ycomb,mark=*,mark options={xscale=1/\scalex,yscale=1/\scaley,scale=0.75},domain=\minn:\maxn,samples=\maxn-\minn+1] plot function{\step(x)*(0.5**x)};
  \end{scope}
  \htick{1*\scaley} node[pos=0.5,right] {$1$};
  %\htick{4*\scaley} node[pos=0.5,left] {$4$};
  %\htick{8*\scaley} node[pos=0.5,left] {$8$};
  %\htick{16*\scaley} node[pos=0.5,left] {$16$};
\end{tikzpicture}
 \\ \vspace{0.5cm}
\begin{tikzpicture}
  \def\scaley{3}
  \begin{scope}[xscale=\scalex,yscale=\scaley]
    %\draw[very thin,color=gray] (-0.1,-1.1) grid (3.9,3.9);
    \draw[->] (\minn-0.5,0) -- (\maxn+0.5,0) node[above] {$n$};
    \draw[->] (0,\ymin/\scaley) -- (0,\ymax/\scaley) node[left] {$(\tfrac{3}{2})^n u_{-n}$};
    %\draw[color=black] plot[id=x] function{1/x^2} 
    %    node[right] {$f(t) = t^{-2}$};
        \draw[color=black,thick,ycomb,mark=*,mark options={xscale=1/\scalex,yscale=1/\scaley,scale=0.75},domain=\minn:\maxn,samples=\maxn-\minn+1] plot function{\step(-x)*((3.0/2)**x)};
    % \draw[color=black] plot[id=exp] function{0.05*exp(x)} 
    %    node[right] {$f(t) = \frac{1}{20} e^t$};
  \end{scope}
  \htick{1*\scaley} node[pos=0.5,right] {$1$};
\end{tikzpicture} \;\;
\begin{tikzpicture}
   \def\scaley{1.5}
  \begin{scope}[xscale=\scalex,yscale=\scaley]
    %\draw[very thin,color=gray] (-0.1,-1.1) grid (3.9,3.9);
    \draw[->] (\minn-0.5,0) -- (\maxn+0.5,0) node[above] {$n$};
    \draw[->] (0,\ymin/\scaley) -- (0,\ymax/\scaley) node[left] {$(\tfrac{1}{2})^n u_n + (\tfrac{3}{2})^n u_{-n}$};
    \draw[color=black,thick,ycomb,mark=*,mark options={xscale=1/\scalex,yscale=1/\scaley,scale=0.75},domain=\minn:\maxn,samples=\maxn-\minn+1] plot function{\step(x)*(0.5**x) + \step(-x)*((3.0/2)**x)};
  \end{scope}
  \htick{2*\scaley} node[pos=0.5,right] {$2$};
  \htick{1*\scaley} node[pos=0.5,right] {$1$};
  %\htick{4*\scaley} node[pos=0.5,left] {$4$};
  %\htick{8*\scaley} node[pos=0.5,left] {$8$};
  %\htick{16*\scaley} node[pos=0.5,left] {$16$};
\end{tikzpicture}
\caption{Real valued sequences.  The top left plot shows the step sequence $u$.} \label{fig:realvaluedsequencestepetc}
\end{figure}
}

{
\def\xmax{1.75}
\def\ymax{1.75}
\def\zscale{1.5}
\newcommand{\graybox}{\draw [draw=none,fill=lightgray] (-\xmax,-\ymax) rectangle (\xmax,\ymax);}
\newcommand{\drawzplane}{
  \draw [<->] (-\xmax,0) -- (\xmax,0) node [above left]  {$\Re$};
  \draw [<->] (0,-\ymax) -- (0,\ymax) node [below right] {$\Im$};
  \draw [dashed] (0,0) circle (1);
}
\begin{figure}[tp]
  \centering
  \begin{tikzpicture}[scale=\zscale]
    \path [draw=none,fill=lightgray] (0,0) circle (1) ;
    \drawzplane
  \end{tikzpicture}
  \hspace{0.3cm}
  \begin{tikzpicture}[scale=\zscale]
    \path [draw=none,fill=lightgray] (0,0) circle (0.5);
    \drawzplane
    \begin{scope}[yscale=1/\zscale]
      \vtick{0.5} node[pos=0.5,below] {$\tfrac{1}{2}$};
    \end{scope}
  \end{tikzpicture} 
  \\ \vspace{0.3cm}
  \begin{tikzpicture}[scale=\zscale]
    \graybox
    \path [draw=none,fill=white] (0,0) circle (3/2);
    \drawzplane
     \begin{scope}[yscale=1/\zscale]
      \vtick{-3/2} node[pos=0.5,below] {$-\tfrac{3}{2}$};
    \end{scope}
  \end{tikzpicture}
  \;\;\;
  \begin{tikzpicture}[scale=\zscale]
    \graybox
    \path [draw=none,fill=white] (0,0) circle (3/2);
    \path [draw=none,fill=lightgray] (0,0) circle (0.5);
    \drawzplane
    \begin{scope}[yscale=1/\zscale]
      \vtick{-3/2} node[pos=0.5,below] {$-\tfrac{3}{2}$};
      \vtick{0.5} node[pos=0.5,below] {$\tfrac{1}{2}$};
    \end{scope}
  \end{tikzpicture}
  \caption{Regions of convergence of the z-transforms of (unshaded) the step sequence $u$ (top left), the sequence $(\tfrac{1}{2})^{n}u_n$ (top right), the sequence $(\tfrac{3}{2})^{n} u_{-n}$ (bottom left), and the sequence $(\tfrac{1}{2})^{n}u_n + (\tfrac{3}{2})^{n} u_{-n}$ (bottom right).  The unit circle is indicated by the dashed circle.  The region of convergence takes the form of the complex plane with a disc an origin centered disc removed (top), a disc with the origin removed (bottom left), an annulus (bottom right), the entire complex plane, or the empty set. }\label{fig:rocexampleztrans1}
\end{figure}
}

\newcommand{\stirling}[2]{\genfrac{[}{]}{0pt}{}{#1}{#2}}
\newcommand{\eulerian}[2]{\genfrac{\langle}{\rangle}{0pt}{}{#1}{#2}}

Given the z-transform $\calZ c$ the sequence $c$ can be recovered by the inverse z-transform
\[
c_n = \frac{1}{2\pi j} \oint_C \calZ c(z) z^{n-1} dz
\]
where $C$ is a counterclockwise closed path encircling the origin and within the region of convergence $\roc_z c$.  Similarly to the inverse Laplace transform (Section~\ref{sec:laplace-transform}), direct calculation of the inverse z-transform requires a form of integration called \term{contour integration} that we will not consider here~\citep{Stewart_ComplexAnalysis_2004}.  For our purposes, and for many engineering purposes, it suffices to remember only the following z-transform pair 
\[
\calZ\big( [n]_k u_n \big) = \frac{k! z }{(z - 1)^{k+1}} \qquad \abs{z} > 1 \qquad \text{(Exercise~\ref{exer:fallingfacztransform})}
\]
where
\[
[n]_k = n (n-1) \dots (n-k+1)
\]
is called the~\term{falling factorial}~\cite[p. 48]{concretemath_1994}.  In the case that $k=0$ the falling factorial is defined as $[n]_0 = 1$ for all $n\in\ints$.  

Let $a \in \complex$.  The z-transforms of the sequence $c_n$ and the sequence with $n$th element $a^nc_n$ are related by 
\[
\calZ(a^n c_n)(z) = \sum_{n \in \ints} a^n c_n z^{-n} = \sum_{n \in \ints} c_n (z/a)^{-n} = \calZ c(z/a) \qquad \frac{z}{a} \in \roc_z c.
\]
This is called the \term{scaling} property of the z-transform.  The region of convergence of $\calZ(a^n c_n)$ is the set of complex numbers $z$ such that $z/a \in \roc_z c$.  Using this property the z-transform of the sequence $a^n [n]_k u_n$ is
\begin{equation}\label{eeq:onlyztrans}
\calZ\big( a^n [n]_k u_n \big) = \frac{k! a^k z }{(z - a)^{k+1}} \qquad \abs{z} > \abs{a}.
\end{equation}
This is the only z-transform pair we require here.  We will have particular use of the case when $k$ is $0$ or $1$.  In the case that $k=0$ we obtain the z-transform pair
\[
\calZ\big( a^n u_n \big) = \frac{ z }{z - a} \qquad \abs{z} > \abs{a}
\]
and in the case that $k=1$ we obtain
\[
\calZ\big( a^n n u_n \big) = \frac{ a z }{(z - a)^{2}} \qquad \abs{z} > \abs{a}.
\]

The z-transform of a sequence $c$ and of the shifted sequence with $n$th element $c_{n-\ell}$ where $\ell \in \ints$ are related by
\begin{align}
\calZ(c_{n-\ell})(z) &= \sum_{n \in \ints} c_{n-\ell} z^{-n} \nonumber \\
&= \sum_{n \in \ints} c_{n} z^{-(n+\ell)} \nonumber \\
&= z^{-\ell} \sum_{n \in \ints} c_n z^{-n} = z^{-\ell} \calZ c(z) \qquad z \in \roc_z c. \label{eq:timeshiftpropertyztransform}
\end{align}
This is called the \term{shifting} property of the z-transform.   % The time-shift property can also be discovered by consideration of the discrete time system $H(x) = T_{P\ell}(G(x))$ formed by the composition of the time shifter $T_{P\ell}$ and the discrete time system $G$ with period $P$ and impulse response $g$ having region of convergence $R_g$.  If $h$ is the discrete impulse response of $H$ then
% \begin{align*}
% H(x) = T_{P\ell}(G(x)) &= T_{P\ell}\big( \sum_{n \in \ints} g_n T_{Pn}(x) \big) \\
% &= \sum_{n \in \ints} g_{n} T_{P(n+\ell)}(x) \\
% &= \sum_{n \in \ints} g_{n - \ell} T_{Pn}(x) 
% \end{align*}
% and so $h_n = g_{n-\ell}$.  The transfer function of $H$ is
% \[
% \lambda(H) = \lambda(T_{P\ell})\lambda(G) = e^{-sP} \lambda(G) \qquad e^{sP} \in R_g.
% \]
% Puting $z = e^{sP}$ we again obtain the time shift property of the z-transform
% \[
% \calZ(h) = \calZ(g_{n-\ell}) = z^{-\ell} \calZ(g) \qquad z \in R_g.
% \]

The z-transform obeys a convolution property analogous to that of the Laplace transform.  Let $F$ and $G$ be discrete time systems with periods $P$ and discrete impulse responses $f$ and $g$.  Let $H x = F G x$ be the discrete time system formed by the composition $F$ and $G$ and with domain $\dom_P f\,g$.  As shown in Section~\ref{sec:discr-time-impulse} the discrete impulse response of $H$ is the discrete convolution $f * g$.  Recall from~\eqref{eq:composedtransferfunction} that the transfer function of a composition of linear time invariant systems is given by the product of the transfer functions, that is,
\[
\lambda H(s)  = \lambda G(s) \lambda F(s) \qquad e^{sP} \in \roc_z f \cap \roc_z g.
\]
For $s$ such that $e^{sP} \in \roc_z f \cap \roc_z g$ we have
\[
\calZ f(e^{sP}) = \lambda F(s) , \;\;\; \calZ g(e^{sP}) = \lambda G(s), \;\;\; \calZ(f * g)(e^{sP}) = \lambda H(s)
\]
from which it follow that
\[
\calZ( f * g)(z) = \calZ f(z) \calZ g(z) \qquad z \in \roc_z f \cap \roc_z g.
\]
That is, the z-transform of a convolution of sequences is the multiplication of the z-transforms of those sequences.  The region of convergence of $\calZ(f*g)$ is the intersection of the regions of convergence of $\calZ f$ and $\calZ g$.  This is called the \term{convolution property} of the z-transform.
%BLERG needs region of convergence done better.

Let $H$ be a stable discrete time system with absolutely summable discrete impulse response $h$.  Because $h$ is absolutely summable we have that $\roc_z h$ contains the complex unit circle, that is, $\roc_z h$ contains all those complex numbers with magnitude equal to one.  We have the following relationships between the transfer function, the spectrum, the discrete time Fourier transform, and the z-transform of a stable discrete time system $H$ and its discrete impulse response $h$,
\[
\lambda H(j2\pi f) = \Lambda H(f) = \calD h(f) = \calZ h(e^{2\pi j P f}).
\]

\section{Difference equations}\label{sec:difference-equations}

\nocite{SolimanAndSrinath_1990}

We have previously shown that interesting systems are found by consideration of a linear differential equation with constant coefficients.  We have used these systems to model electrical and mechanical devices (Chapter~\ref{sec:syst-modell-diff}).  We will find that interesting discrete time systems are found by consideration of a linear~\term{difference equation} with constant coefficients.  That is, an equation relating two sequences $c$ and $d$ of the form
\begin{equation}\label{eq:differenceequform}
\sum_{\ell=0}^{m} a_\ell c_{n - \ell} = \sum_{\ell=0}^{k} b_\ell d_{n - \ell} \qquad n \in \ints
\end{equation}
where $a_0,\dots,a_m$ and $b_0,\dots,b_k$ are complex constants.  For example, in Section~\ref{sec:fast-four-transf} we found that the number of complex operations required to compute a Cooley-Tukey fast Fourier transform of size $N = 2^n$ was $C_N = C_{2^n} = d_n$ where the sequence $d_n$ satisfied the equation~\eqref{eq:recursiveeqcomplexityfft}
\[
2^{n+1} = d_n - 2 d_{n-1} \qquad n \geq 0
\]
and where $C_1 = d_0 = 0$.  This is in the form of~\eqref{eq:differenceequform} if we suppose that $d_n = 0$ for $n \leq 0$ and put $c_n = 2^{n+1} u_{n-1}$.

Another example is the Fibonacci sequence
\[
0,1,1,2,3,5,8,13,21,\dots.
\]
Each element of the Fibonnacci sequence after the second is given by the sum of the previous two elements.  Letting $d_n$ be the elements of the Fibonacci sequence so that $d_0 = 0, d_1 = 1, d_2 = 1, \dots$ and putting $d_n = 0$ for $n < 0$ we have
\[
d_{n} - d_{n-1} - d_{n-2} = \delta_{n-1}.
\] 
This is in the form~\eqref{eq:differenceequform} if we put $c_n = \delta_{n-1}$.

In order to study difference equations it is useful to study the equation
\begin{equation}\label{eq:differenceequformsystem}
\sum_{\ell=0}^{m} a_\ell T_{P\ell} x = \sum_{\ell=0}^{k} b_\ell T_{P\ell} y
\end{equation}
that relates two signals $x$ and $y$.  \citet[Sec.~9.5]{Zemanian_dist_theory_1965} calls~\eqref{eq:differenceequformsystem} the \term{continuous variable case} of a linear difference equation with constant coefficients.  If $x$ and $y$ are signals satisfying this equation then the samples of $x$ and $y$ at multiples of $P$ satisfy~\eqref{eq:differenceequform}.  That is, if we define sequences $c$ and $d$ by $c_n = x(nP)$ and $d_n = y(nP)$ then $c$ and $d$ satisfy~\eqref{eq:differenceequform} whenever $x$ and $y$ satisfy~\eqref{eq:differenceequformsystem}.

Suppose that $H$ is a linear shift-invariant system with the property that the response $y = H x$ to input signal $x$ is such that $x$ and $y$ satisfy~\eqref{eq:differenceequformsystem}.  The response of $H$ to the complex exponential signal $e^{st}$ satisfies $He^{st} = \lambda H(s) e^{st}$.  Substituting $x(t) = e^{st}$ and $y = \lambda H(s) e^{st}$ into~\eqref{eq:differenceequformsystem} gives
\[
\sum_{\ell=0}^{m} a_\ell T_{P\ell} e^{st} = \sum_{\ell=0}^{k} b_\ell T_{P\ell} (\lambda e^{st})
\]
where, to simplify notation, we have written simply $\lambda$ for $\lambda H(s)$ above.  Since $T_{P\ell} e^{st} = e^{-sP\ell} e^{st}$ we find that
\[
e^{st} \sum_{\ell=0}^{m} a_\ell e^{-sP\ell}  = \lambda e^{st} \sum_{\ell=0}^{k} b_\ell e^{-sP\ell}
\]
and rearranging we find that the transfer function $\lambda H$ satisfies
\[
\lambda H(s) = \frac{\sum_{\ell=0}^{m} a_\ell e^{-sP\ell}}{\sum_{\ell=0}^{k} b_\ell e^{-sP\ell}} = \frac{\sum_{\ell=0}^{m} a_\ell z^{-\ell}}{\sum_{\ell=0}^{k} b_\ell z^{-\ell}} = z^{k-m} \frac{\sum_{\ell=0}^{m} a_\ell z^{m-\ell}}{\sum_{\ell=0}^{k} b_\ell z^{k-\ell}}
\]
where $z = e^{sP}$.  Suppose that $h$ is a sequence with z-transform
\[
\calZ h(z) = \lambda H(s) = z^{k-m} \frac{\sum_{\ell=0}^{m} a_\ell z^{m-\ell}}{\sum_{\ell=0}^{k} b_\ell z^{k-\ell}}.
\]
It follows from~\eqref{eq:reltransferfunztrans} that $H$ is a discrete time system with discrete impulse response $h$.  By applying the inverse z-transform we can find an explicit expression for $h$.  This procedure is similar to how the impulse response of a system described by a differential equation was found by application of the inverse Laplace transform in Section~\ref{sec:poles-zeros-stab}.  In the case that $m > k$ the term $z^{k-m}$ can be incorporated into the denominator obtaining
\[
\calZ h(z) = \frac{a_0z^m + a_1 z^{m-1} + \dots + a_m}{b_0z^m + b_1 z^{k-1} + \dots + b_k z^{m-k}}
\]
and in the case that $m < k$ the term $z^{m-k}$ can be incorporated into the numerator obtaining
\[
\calZ h(z) = \frac{a_0z^k + a_1 z^{k-1} + \dots + a_m z^{k-m}}{b_0z^k + b_1 z^{k-1} + \dots + b_k}.
\] 
In either case the order of the polynomials on the numerator and denominator are the same, that is, the order is $w = \max(m,k)$.

By factorising the polynomials on the numerator and denominator we obtain
\[
\calZ h(z) = \frac{a_0}{b_0} \frac{(z-\alpha_0)(z - \alpha_1)\cdots(z - \alpha_{w})}{(z-\beta_0)(z - \beta_1)\cdots(z - \beta_{w})}
\]
where $\alpha_0, \dots, \alpha_w$ are the roots of the numerator polynomial and $\beta_0, \dots, \beta_w$ are the roots of the denominator polynomial.  If the numerator and denominator polynomials share one or more roots, then these roots cancel leaving the simpler expression
\begin{equation}\label{eq:ztransfuncpoleszeros}
\calZ h(z) = \frac{a_0}{b_0} \frac{(z-\alpha_d)(z - \alpha_{d+1})\cdots(z - \alpha_{w})}{(z-\beta_d)(z - \beta_{d+1})\cdots(z - \beta_{w})},
\end{equation}
where $d$ is the number of shared roots, these shared roots being 
\[
\alpha_0 = \beta_0, \;\; \alpha_1 = \beta_1, \;\; \dots, \;\;  \alpha_{d-1} = \beta_{d-1}.
\]
The roots from the numerator $\alpha_d, \dots, \alpha_w$ are called the \term{zeros} and the roots from the denominator $\beta_d, \dots, \beta_w$ are called the \term{poles}.  For a discrete time system, the number of poles and zeros are equal.   A pole-zero plot is constructed by marking the complex plane with a cross at the location of each pole and a circle at the location of each zero (Figure~\ref{fig:polezeroplotdisctime}). %If $k>m$ then there exist $k-m$ zeros located at the origin.  If $k<m$ there exist $m-k$ poles located at the origin.  Otherwise, if $k=m$ there are no extra poles or zeros.

{
\def\xmax{1.75}
\def\ymax{1.75}
\def\zscale{1.5}
\def\sqrttwo{0.707107}
\newcommand{\graybox}{\draw [draw=none,fill=lightgray] (-\xmax,-\ymax) rectangle (\xmax,\ymax);}
\newcommand{\drawzplane}{
  \draw [<->] (-\xmax,0) -- (\xmax,0) node [above left]  {$\Re$};
  \draw [<->] (0,-\ymax) -- (0,\ymax) node [below right] {$\Im$};
  \draw [dashed] (0,0) circle (1);
}
\begin{figure}[p]
  \centering
  \begin{tikzpicture}[scale=\zscale]
    \poletikz{2.0/3}{0}; %\node[below] at (2.0/3,0) {$\tfrac{2}{3}$};
    \zerotikz{0}{0} \node[below] at (0,0) {};
    %\path [draw=none,fill=lightgray] (0,0) circle (1) ;
    \drawzplane
  \end{tikzpicture}
  \hspace{0.3cm}
  \begin{tikzpicture}[scale=\zscale]
    \poletikz{-3.0/2.0}{0}; % \node[below] at (-3.0/2,0) {$-\tfrac{3}{2}$};
    \zerotikz{0}{0} \node[below] at (0,0) {};
    %\path [draw=none,fill=lightgray] (0,0) circle (0.5);
    \drawzplane
    \begin{scope}[yscale=1/\zscale]
      %\vtick{0.5} node[pos=0.5,below] {$\tfrac{1}{2}$};
    \end{scope}
  \end{tikzpicture} 
  \\ \vspace{0.3cm}
  \begin{tikzpicture}[scale=\zscale]
    \zerotikz{0}{0} \node[below] at (0,0) {};
    \node[above right,font=\tiny] at (0,0) {$2$};
    \poletikz{2/3}{1/3};
    \poletikz{2/3}{-1/3};
    %\path [draw=none,fill=white] (0,0) circle (3/2);
    \drawzplane
     \begin{scope}[yscale=1/\zscale]
      %\vtick{-3/2} node[pos=0.5,below] {$-\tfrac{3}{2}$};
    \end{scope}
  \end{tikzpicture}
  \;\;\;
  \begin{tikzpicture}[scale=\zscale]
    \zerotikz{0}{0} \node[below] at (0,0) {};
    \node[above right,font=\tiny] at (0,0) {$2$};
    \poletikz{-\sqrttwo}{\sqrttwo};
    \poletikz{-\sqrttwo}{-\sqrttwo};
    %\path [draw=none,fill=white] (0,0) circle (3/2);
    %\path [draw=none,fill=lightgray] (0,0) circle (0.5);
    \drawzplane
    \begin{scope}[yscale=1/\zscale]
      %\vtick{-3/2} node[pos=0.5,below] {$-\tfrac{3}{2}$};
      %\vtick{0.5} node[pos=0.5,below] {$\tfrac{1}{2}$};
    \end{scope}
  \end{tikzpicture}
  \caption{Pole zero plots for discrete time systems corresponding with first order difference equations $c_n = d_n - \tfrac{2}{3} d_{n-1}$ (top left) and $c_n = d_n + \tfrac{3}{2} d_{n-1}$ (top right) and second order difference equations $z^2 - \tfrac{4}{3} x - \tfrac{4}{9}$ (bottom left) and $z^2 + \sqrt{2}z + 1$ (bottom right).  The plots of the left correspond with stable systems because all poles are contained inside the complex unit circle (dashed).  Plots on the right correspond with unstable systems because there exist poles on or outside the complex unit circle.  The small $2$'s above the zero on the lower plots indicate the existence of two zeros at the origin.}\label{fig:polezeroplotdisctime}
\end{figure}
}

The z-transform pair~\eqref{eeq:onlyztrans} has the term $z$ on its numerator and so it is convenient to write
\[
\calZ h(z) = \frac{a_0}{b_0} z \frac{(z-\alpha_d)(z - \alpha_{d+1})\cdots(z - \alpha_{w})}{z(z-\beta_d)(z - \beta_{d+1})\cdots(z - \beta_{w})}.
\]
Applying partial fraction to polynomial quotient yields
\[
\calZ h(z) = \frac{a_0}{b_0} z \sum_{\ell \in K} \frac{A_\ell }{(z - \beta_\ell)^{r_\ell}}
\] 
where $r_\ell$ are positive integers, $A_\ell$ are complex constants, and $K$ is a subset of the indices from $\{d,d+1,\dots,w\}$.  We need to consider those terms where $\beta_\ell = 0$ separately.  Let $K_1$ be the subset of indices from $K$ such that $\beta_\ell = 0$ when $\ell \in K_1$ and let $K_2$ be the subset such that $\beta_\ell \neq 0$ when $\ell \in K_2$.  Now 
\[
\calZ h(z) = \frac{a_0}{b_0} \sum_{\ell \in K_1} \frac{A_\ell }{z^{r_\ell-1}} + \sum_{\ell \in K} B_\ell \frac{  \beta_\ell^{r_\ell-1} (r_\ell-1)! z }{(z - \beta_\ell)^{r_\ell}}.
\] 
where
\[
B_\ell = \frac{a_0 A_\ell}{b_0 \beta_\ell^{r_\ell-1} (r_\ell-1)! }.
\] 
Those terms of the form $A_\ell z^{1 - r_\ell}$ correspond with sequences $A_\ell \delta_{n+r_\ell-1}$ where $\delta$ is the delta sequence.  From~\eqref{eeq:onlyztrans} with $k = r_\ell - 1$ those terms of the form 
\[
\frac{\beta_\ell^{r_\ell-1} (r_\ell-1)! z }{(z - \beta_\ell)^{r_\ell}}
\]
are found to correspond with sequences $B_\ell \beta_\ell^n [n]_{r_\ell-1} u_n$ where $u$ is the step sequence.  Other sequences with the same z-transform are disregarded because they are not right sided and so do not correspond with a causal discrete time system.  Combing the above results we find that the discrete impulse response $h$ of the discrete time system $H$ takes the form
\[
h_n = \frac{a_0}{b_0} \sum_{\ell \in K_1} A_\ell \delta_{n+r_\ell-1} + \sum_{\ell \in K_2} B_\ell \beta_\ell^n [n]_{r_\ell-1} u_n.
\]
The discrete impulse response is absolutely summable only if the poles satisfy $\abs{\beta_\ell} < 1$ for all $\ell = d,\dots,w$ as a result of the terms $\beta_\ell^n$ that occur when $\beta_\ell \neq 0$.  The system $H$ is stable if and only if $h$ is absolutely summable (Exercise~\ref{excer:stableimpulserespdiscretetime}) and so a discrete time system is stable if and only if no poles lie outside or on the complex unit circle.

We now consider some specific examples of difference equations and their corresponding discrete time systems.  Consider the difference equation
\begin{equation}\label{eq:firstorderdifferenceequation}
c_n = d_n - a d_{n-1} \qquad n \in \ints
\end{equation}
where $a \in \complex$.  This is called a \term{first order difference equation}.  Suppose that $H$ is a discrete time system such that the response $y = H(x)$ to input $x$ satisfies
\[
x = y - a T_{P}(y).
\]
The transfer function of $H$ is
\[
\lambda(H,s) = \frac{1}{1 - a e^{-sP}} = \frac{1}{1 - a z^{-1}} = \frac{z}{z - a}
\]
where $z = e^{sP}$.  The system has a single zero at $z=0$ and a single pole at $z = a$.  The system will be stable if and only if this pole lies strictly inside the complex unit circle, that is, if and only if $\abs{a} < 1$.  The discrete impulse response is found to be $h_n = a^n u_n$ by putting $k=0$ in~\eqref{eeq:onlyztrans}.  Other sequences with this z-transform are discarded because they do not correspond with a causal system.  %The impulse response is absolutely summable if and only if when $\abs{a}< 1$ and so the discrete time system $H$ is stable if and only if $\abs{a} < 1$.  
When $\abs{a} < 1$ the region of convergence contains the unit circle and the system has spectrum
\[
\Lambda(H,f) = \lambda(H,j2\pi f) = \calZ(h, e^{2\pi j P f}) = \frac{e^{2\pi j P f}}{e^{2\pi j P f} - a}.
\]
The magnitude and phase spectrum are plotted in Figure~\ref{fig:spectrumdiscresystem1} in the case that $a = \tfrac{1}{2}$ and $\tfrac{1}{10}$.

Now consider the difference equation
\[
c_n = d_n - a d_{n-1} - b d_{n-2} \qquad n \in \ints.
\]
where $a,b \in \complex$.  This is called a~\term{second order difference equation}.  Suppose that $H$ is a discrete time system with reponse $y=H(x)$ satisfying the equation $x = y - aT_{P}(y) - b T_{2P}(y)$.  The transfer function is
\[
\lambda(H) = \frac{1}{1 - a e^{-sP} - b e^{-2sP}} = \frac{z^2}{z^2 - a z - b} = \calZ(h)
\]
where $h$ is the discrete impulse response of $H$.  The system has two zeros at $z = 0$ and two poles given by the roots of the polynomial $z^2 - a z - b$.  The z-transform can be inverted to obtain $h$ (Exercise~\ref{exer:findimpulseresponsesecondorderdiscrete}).  The system $H$ is stable if and only if both poles lie strictly inside the complex unit circle (Figure~\ref{fig:polezeroplotdisctime}).  In this case, $H$ has spectrum
\[
\Lambda(H,f) = \calZ(h, e^{2\pi j P f}) = \frac{e^{2\pi jP f}}{e^{4\pi jP f} - a e^{2\pi jP f} - b}.
\]
%The spectrum is plotted in ...

\begin{figure}[tp]
\centering
\def\P{0.5}
\def\F{1/\P}
  \begin{tikzpicture}[domain=-5:5,samples=200]
    % \draw[very thin,color=gray] (-0.1,-1.1) grid (3.9,3.9);
    \begin{scope}[yscale=2]
      \draw[->] (-5.5,0) -- (5.5,0) node[above] {$f$}; 
      \draw[->] (0,-0.3) -- (0,2.5) node[right] {$\abs{\Lambda(H, f)}$};
      % \draw[smooth,color=black,thick] plot function{sin(3.14159265359*x)*exp(-x)};
      \draw[thick] plot[] file {data/discretetimesystem/datafftcomplexityabs2.csv};
      \node[above] at (-\F,2) {$\tfrac{1}{2}$};
      %\draw[thick] plot[] file {data/discretetimesystem/datafftcomplexityabs4.csv};
      \draw[thick] plot[] file {data/discretetimesystem/datafftcomplexityabs10.csv};
      \node at (-\F,0.95) {$\tfrac{1}{10}$};
      % \node at (9.3,-1) {$0$};
      % \htick{2} node[pos=0.5,left] {$2$};
      \htick{2} node[pos=0.5,above left] {$2$};
    \end{scope}
    \vtick{-\F} node[pos=0.5,below] {$-F$};
    \vtick{\F} node[pos=0.5,below] {$F$};
  \end{tikzpicture}
\medskip
  \begin{tikzpicture}[domain=-5:5,samples=200]
    % \draw[very thin,color=gray] (-0.1,-1.1) grid (3.9,3.9);
    \begin{scope}[yscale=3]
      \draw[->] (-5.5,0) -- (5.5,0) node[above] {$f$};
      \draw[->] (0,-0.75) -- (0,0.75) node[right] {$\angle{\Lambda(H, f)}$};
      % \draw[smooth,color=black,thick] plot function{sin(3.14159265359*x)*exp(-x)};
      \draw[thick] plot[] file {data/discretetimesystem/datafftcomplexityangle2.csv};
      \node[above] at (-\F-0.333,pi/6) {$\tfrac{1}{2}$};
      %\draw[thick] plot[] file {data/discretetimesystem/datafftcomplexityangle4.csv};
      \draw[thick] plot[] file {data/discretetimesystem/datafftcomplexityangle10.csv};
      \node at (-\F-0.5,0.2) {$\tfrac{1}{10}$};
      % \node at (9.3,-1) {$0$};
      % \htick{2} node[pos=0.5,left] {$2$};
      \htick{pi/6} node[pos=0.5,right] {$\tfrac{\pi}{6}$};
      \htick{-pi/6} node[pos=0.5,left] {$-\tfrac{\pi}{6}$};
      %\htick{2*pi} node[pos=0.5,left] {$2\pi$};
      %\htick{pi/2} node[pos=0.5,left] {$\tfrac{\pi}{2}$};
      %\htick{3*pi/2} node[pos=0.5,left] {$\tfrac{3\pi}{2}$}; 
      %\htick{-3.14159265359} node[pos=0.5,left] {$-\pi$};
    \end{scope}
    %\vtick{-\F} node[pos=0.5,below] {$-F$};
    \vtick{\F} node[pos=0.5,above right] {$F$};
  \end{tikzpicture}
\caption{Magnitude and phase spectrum of the discrete time $H$ with discrete implulse response $h_n = a^nu_n$ for $a=\tfrac{1}{2}$ and $\tfrac{1}{10}$ and period $P = \tfrac{1}{F}$.  The spectrum is periodic with period $F = \tfrac{1}{P}$.  This system corresponds with the first order difference equation $c_n = d_n - 2 d_{n-1}$.} \label{fig:spectrumdiscresystem1}
\end{figure}

% \begin{figure}[tp]
% \centering
% \def\P{0.5}
% \def\F{1/\P}
%   \begin{tikzpicture}[domain=-5:5,samples=200]
%     % \draw[very thin,color=gray] (-0.1,-1.1) grid (3.9,3.9);
%     \begin{scope}[yscale=2]
%       \draw[->] (-5.5,0) -- (5.5,0) node[above] {$f$}; 
%       \draw[->] (0,-0.3) -- (0,2.5) node[right] {$\abs{\Lambda(H, f)}$};
%       % \draw[smooth,color=black,thick] plot function{sin(3.14159265359*x)*exp(-x)};
%       \draw[thick] plot[] file {data/discretetimesystem/secondorderlowpassabs.csv};
%       \node[above] at (-\F,2) {$\tfrac{1}{2}$};
%       \draw[thick] plot[] file {data/discretetimesystem/secondorderbandpassabs.csv};
%       \node at (-\F,0.95) {$\tfrac{1}{10}$};
%       % \node at (9.3,-1) {$0$};
%       % \htick{2} node[pos=0.5,left] {$2$};
%       \htick{2} node[pos=0.5,above left] {$2$};
%     \end{scope}
%     \vtick{-\F} node[pos=0.5,below] {$-F$};
%     \vtick{\F} node[pos=0.5,below] {$F$};
%   \end{tikzpicture}
% \medskip
%   \begin{tikzpicture}[domain=-5:5,samples=200]
%     % \draw[very thin,color=gray] (-0.1,-1.1) grid (3.9,3.9);
%     \begin{scope}[yscale=3]
%       \draw[->] (-5.5,0) -- (5.5,0) node[above] {$f$};
%       \draw[->] (0,-0.75) -- (0,0.75) node[right] {$\angle{\Lambda(H, f)}$};
%       % \draw[smooth,color=black,thick] plot function{sin(3.14159265359*x)*exp(-x)};
%       \draw[thick] plot[] file {data/discretetimesystem/secondorderlowpassangle.csv};
%       \node[above] at (-\F-0.333,pi/6) {$\tfrac{1}{2}$};
%        \draw[thick] plot[] file {data/discretetimesystem/secondorderbandpassangle.csv};
%       \node at (-\F-0.5,0.2) {$\tfrac{1}{10}$};
%       % \node at (9.3,-1) {$0$};
%       % \htick{2} node[pos=0.5,left] {$2$};
%       \htick{pi/6} node[pos=0.5,right] {$\tfrac{\pi}{6}$};
%       \htick{-pi/6} node[pos=0.5,left] {$-\tfrac{\pi}{6}$};
%       %\htick{2*pi} node[pos=0.5,left] {$2\pi$};
%       %\htick{pi/2} node[pos=0.5,left] {$\tfrac{\pi}{2}$};
%       %\htick{3*pi/2} node[pos=0.5,left] {$\tfrac{3\pi}{2}$}; 
%       %\htick{-3.14159265359} node[pos=0.5,left] {$-\pi$};
%     \end{scope}
%     %\vtick{-\F} node[pos=0.5,below] {$-F$};
%     \vtick{\F} node[pos=0.5,above right] {$F$};
%   \end{tikzpicture}
% \caption{Magnitude and phase spectrum of the discrete time $H$ for corresponds with the second order difference equation $c_n = d_n - 2 d_{n-1}$$a=\tfrac{1}{2}$ and $\tfrac{1}{10}$ and period $P = \tfrac{1}{F}$.  The spectrum is periodic with period $F = \tfrac{1}{P}$.  This system corresponds with the first order difference equation $c_n = d_n - 2 d_{n-1}$.} \label{fig:spectrumdiscresystem2}
% \end{figure}



%\begin{comment}




\section{Multirate filters} \label{sec:multirate-filters}


%\end{comment}




%\end{comment}



% \chapter{Sampling and interpolation} \label{sec:sampl-interp}

% %In each of the tests conducted we have made use of the computer sound card

% Let $x$ be a signal with Fourier transform $\hat{x} = \calF(x)$ and let
% \begin{equation}\label{eq:periodizedhatxp}
% \hat{x}_p(f) = \sum_{m \in \ints} \hat{x}(f - m).
% \end{equation}
% The signal $\hat{x}_p$ is periodic with period one since for every integer $k$,
% \[
% \hat{x}_p(f - k) = \sum_{m \in \ints} \hat{x}(f - k - m) = \sum_{m \in \ints} \hat{x}(f - m) = \hat{x}_p(f).
% \]
% For this reason $\hat{x}_p$ is sometimes called the \term{periodised} or \term{wrapped} version of $\hat{x}$~\citep{Fisher_tsa_cd_1994}.  We plot functions $\hat{x}$ and their periodised versions $\hat{x}_p$ in Figure~\ref{fig:periodisedfunctions}.

% Assume that we can write the periodic signal $\hat{x}_p(f)$ as a series
% \begin{equation}\label{eq:hatxpdiscreteft}
% \hat{x}_p(f) = \sum_{n\in\ints} x_n e^{-j2\pi f n}.
% \end{equation}
% The coefficients $x_n$ in this series can be recovered by
% \begin{equation}\label{eq:idfftxnhatxp}
% x_n = \int_{-1/2}^{1/2} \hat{x}_p(f) e^{2\pi jfn} df.
% \end{equation}
% To see this write
% \begin{align*}
% \int_{-1/2}^{1/2} \hat{x}_p(f) e^{2\pi jfn} df &= \int_{-1/2}^{1/2} \big( \sum_{m\in\ints} x_m e^{-j2\pi f m} \big) e^{2\pi jfn} df \\
% &= \sum_{m\in\ints} x_m   \int_{-1/2}^{1/2} e^{-j2\pi fm}e^{j2\pi fn} df \\
% &= \sum_{m\in\ints} x_m   \int_{-1/2}^{1/2} e^{j2\pi f(n-m)} df \\
% &= \sum_{m\in\ints} x_m   \sinc(n-m) \\
% &= x_n
% \end{align*}
% because $\sinc(n-m) = 1$ when $n = m$ and zero otherwise.  The periodic function $\hat{x}_p$ is called the \term{discrete Fourier transform} of the sequence $x_n$. 

% % To see that write
% % \begin{align*}
% % \hat{x}_p(f) &= \sum_{n\in\ints} \int_{-1/2}^{1/2} \hat{x}_p(\gamma) e^{2\pi j \gamma n} d\gamma e^{-j2\pi f n} \\
% % = \int_{-1/2}^{1/2} \hat{x}_p(\gamma) \sum_{n\in\ints} e^{2\pi j \gamma n} e^{-j2\pi f n} d\gamma\\
% % \end{align*}
% % The periodised signal $\hat{x}_p$ is called the \term{discrete Fourier transform} of the sequence $x_n$.

% Substituting~\eqref{eq:periodizedhatxp} into~\eqref{eq:idfftxnhatxp} we obtain
% \[
% x_n = \int_{-1/2}^{1/2} \sum_{m \in \ints} \hat{x}(f - m) e^{2\pi jfn} df = \sum_{m \in \ints} \int_{-1/2}^{1/2}  \hat{x}(f - m) e^{2\pi jfn} df.
% \]
% %BLERG: where the exchange of integration and summation can be justified by \term{Lebesgue's dominated convergence theorem}~\cite[page~26]{Rudin_real_and_complex_analysis}.  
% By the change of variable $\gamma = f - m$ we obtain
% \begin{align*}
% x_n &= \sum_{m \in \ints} \int_{- 1/2-m}^{1/2-m}  \hat{x}(\gamma) e^{2\pi j n (\gamma + m)} d\gamma \\
% &= \sum_{m \in \ints} \int_{- 1/2-m}^{1/2-m}  \hat{x}(\gamma) e^{2\pi j n \gamma} d\gamma & \text{(since $e^{2\pi j m} = 1$)} \\
% &= \int_{-\infty}^{\infty}  \hat{x}(\gamma) e^{2\pi j n \gamma} d\gamma \\
% & = \calF^{-1}(\hat{x},n) \\
% &= x(n).
% \end{align*}
% Thus, the sequence $x_n$ corresponds with the signal $x$ sampled at the integers, that is $x_n = x(n)$.

% A signal $x$ is called \term{bandlimited} if there exists a positive real number $b$ such that $\calF(x,f) = 0$ for all $\abs{f} > b$.  For example, the $\sinc$ function is bandlimited with bandwidth $\tfrac{1}{2}$ because its Fourier transform $\calF(\sinc,f) = \rect(f) = 0$ for all $\abs{f} > b$.  The value $b$ is referred to as the \term{bandwidth} of the signal $x$.  If $x$ is bandlimited with bandwidth $b\leq\tfrac{1}{2}$, then $x$ can be recovered from its samples at the integers, that is, $x$ can be recovered from the sequence $x_n$.  To see this, first observe that
% \[
% \rect(f) \hat{x}(f - m) = \begin{cases} 
% \hat{x}(f) & m = 0 \\
% 0 & \text{otherwise}
% \end{cases}
% \] 
% since $\hat{x}(f) = 0$ whenever $\abs{f}\geq \tfrac{1}{2}$.  Now, multiplying $\hat{x}_p(f)$ by the rectangle function gives
% \[
% \rect(f)\hat{x}_p(f) = \sum_{m \in \ints} \rect(f) \hat{x}(f - m) = \hat{x}(f).
% \]
% Now consider the signal
% \[
% \tilde{x}(t) = \sum_{n\in\ints} x_n \sinc(t - n).
% \]
% Taking the Fourier transform on both sides gives
% \begin{align*}
% \calF(\tilde{x}) &= \calF\big( \sum_{n\in\ints} x_n \sinc(t - n) \big) \\
% &= \sum_{n\in\ints} x_n \calF \big( \sinc(t - n) \big) \\
% &= \sum_{n\in\ints} x_n e^{-j2\pi f n}\rect(f) & \text{(time shift property of $\calF$)} \\
% &= \rect(f) \hat{x}_p(f) & \text{(from~\eqref{eq:hatxpdiscreteft})} \\
% &= \hat{x}(f) \\
% &= \calF(x,f).
% \end{align*}
% %BLERG: The interchange of $\calF$ and $\sum_{n\in\ints}$ could be justified if $x_n$ is \term{absolutely summable}, not sure what's required for this to happen.  
% Thus, $\calF(\tilde{x}) = \calF(x)$ and application of the inverse Fourier transform reveals that $\tilde{x} = x$, that is
% \[
% x(t) = \sum_{n\in\ints} x_n \sinc(t - n).
% \]
% If instead of sampling at the integers we sample at rate $F_s$ so that $x_n = x(F_s n)$, then, by a similar argument, we find that $x$ can be recovered as
% \[
% x(t) = \sum_{n\in\ints} x_n \sinc(F_s t - n)
% \]
% provided that $x$ is bandlimited with bandwidth $F_s/2$.  This is called the \term{Nyquist criterion}.

% \begin{figure}[p]
% \centering
% {
%   \def\f(#1){exp(-(#1)*(#1)*4)}
%   \def\scaley{2}
%   \begin{tikzpicture}[yscale=\scaley,domain=-2.4:2.4,samples=100]
%     \draw[->] (-2.8,0) -- (2.8,0) node[above] {$f$};
%     \draw[->] (0,-0.35) -- (0,1.2) node[above] {$\hat{x}(f) = e^{-4 f^2}$};
%     \draw[smooth,color=black,thick] plot function{\f(x)};
%     % \htick{1} node[pos=0.5,above left] {$1$};
%     \begin{scope}[yscale=1/\scaley]
%       \vtick{1} node[pos=0.5,below] {$1$};
%       \vtick{-1} node[pos=0.5,below] {$-1$};
%       \vtick{2} node[pos=0.5,below] {$2$};
%       \vtick{-2} node[pos=0.5,below] {$-2$};
%     \end{scope}
%   \end{tikzpicture} 
%   \;\;
%   \begin{tikzpicture}[yscale=\scaley,domain=-2.4:2.4,samples=100]
%     \draw[->] (-2.8,0) -- (2.8,0) node[above] {$f$};
%     \draw[->] (0,-0.35) -- (0,1.2) node[above] {$\hat{x}_p(f)$};
%     \draw[smooth,color=black,dashed] plot function{\f(x)};
%     \draw[smooth,color=black,dashed] plot function{\f(x+1)};
%     \draw[smooth,color=black,dashed] plot function{\f(x+2)};
%     \draw[smooth,color=black,dashed] plot function{\f(x+3)};
%     \draw[smooth,color=black,dashed] plot function{\f(x-1)};
%     \draw[smooth,color=black,dashed] plot function{\f(x-2)};
%     \draw[smooth,color=black,dashed] plot function{\f(x-3)};
%     \draw[smooth,color=black,thick] plot function{\f(x-3)+\f(x-2)+\f(x-1)+\f(x)+\f(x+1)+\f(x+2)+\f(x+3)};
%     \begin{scope}[yscale=1/\scaley]
%       \vtick{1} node[pos=0.5,below] {$1$};
%       \vtick{-1} node[pos=0.5,below] {$-1$};
%       \vtick{2} node[pos=0.5,below] {$2$};
%       \vtick{-2} node[pos=0.5,below] {$-2$};
%     \end{scope}
%   \end{tikzpicture} 
% }
% \\
% \vspace{1cm}
% {
%   \def\ifthenelset(#1,#2,#3){(#1*#2 + !#1*#3)}
%   \def\step(#1){\ifthenelset(((#1)>0),1,0)} %step function
%   \def\rectangle(#1){(\step(#1+1)-\step(#1-1))}
%   \def\f(#1){\rectangle(#1)*(1+cos(pi*(#1)))}
%   \def\scaley{1}
%   \begin{tikzpicture}[yscale=\scaley,domain=-2.4:2.4,samples=100]
%     \draw[->] (-2.8,0) -- (2.8,0) node[above] {$f$};
%     \draw[->] (0,-0.7/\scaley) -- (0,2.4/\scaley) node[above] {$\hat{x}(f) = \rect(f/2)\big(1 + \cos(\pi f)\big)$};
%     \draw[smooth,color=black,thick] plot function{\f(x)};
%     % \htick{1} node[pos=0.5,above left] {$1$};
%     \begin{scope}[yscale=1/\scaley]
%       \vtick{1} node[pos=0.5,below] {$1$};
%       \vtick{-1} node[pos=0.5,below] {$-1$};
%       \vtick{2} node[pos=0.5,below] {$2$};
%       \vtick{-2} node[pos=0.5,below] {$-2$};
%     \end{scope}
%   \end{tikzpicture} 
%   \;\;
%   \begin{tikzpicture}[yscale=\scaley,domain=-2.4:2.4,samples=100]
%     \draw[->] (-2.8,0) -- (2.8,0) node[above] {$f$};
%     \draw[->] (0,-0.7/\scaley) -- (0,2.4/\scaley) node[above] {$\hat{x}_p(f)$};
%     \draw[smooth,color=black,dashed] plot function{\f(x)};
%     \draw[smooth,color=black,dashed] plot function{\f(x+1)};
%     \draw[smooth,color=black,dashed] plot function{\f(x+2)};
%     \draw[smooth,color=black,dashed] plot function{\f(x+3)};
%     \draw[smooth,color=black,dashed] plot function{\f(x-1)};
%     \draw[smooth,color=black,dashed] plot function{\f(x-2)};
%     \draw[smooth,color=black,dashed] plot function{\f(x-3)};
%     \draw[smooth,color=black,thick] plot function{\f(x-3)+\f(x-2)+\f(x-1)+\f(x)+\f(x+1)+\f(x+2)+\f(x+3)};
%     \begin{scope}[yscale=1/\scaley]
%       \vtick{1} node[pos=0.5,below] {$1$};
%       \vtick{-1} node[pos=0.5,below] {$-1$};
%       \vtick{2} node[pos=0.5,below] {$2$};
%       \vtick{-2} node[pos=0.5,below] {$-2$};
%     \end{scope}
%   \end{tikzpicture} 
% } 
% \\
% \vspace{1cm}
% {
%   \def\ifthenelset(#1,#2,#3){(#1*#2 + !#1*#3)}
%   \def\step(#1){\ifthenelset(((#1)>0),1,0)} %step function
%   \def\rectangle(#1){(\step(#1+0.3333)-\step(#1-0.3333))}
%   \def\f(#1){\rectangle(#1)*(1+cos(3*pi*(#1)))}
%   \def\scaley{1}
%   \begin{tikzpicture}[yscale=\scaley,domain=-2.4:2.4,samples=500]
%     \draw[->] (-2.8,0) -- (2.8,0) node[above] {$f$};
%     \draw[->] (0,-0.7/\scaley) -- (0,2.4/\scaley) node[above] {$\hat{x}(f) = \rect(3f/2)\big(1 + \cos(3 \pi f)\big)$};
%     \draw[smooth,color=black,thick] plot function{\f(x)};
%     % \htick{1} node[pos=0.5,above left] {$1$};
%     \begin{scope}[yscale=1/\scaley]
%       \vtick{1} node[pos=0.5,below] {$1$};
%       \vtick{-1} node[pos=0.5,below] {$-1$};
%       \vtick{2} node[pos=0.5,below] {$2$};
%       \vtick{-2} node[pos=0.5,below] {$-2$};
%     \end{scope}
%   \end{tikzpicture} 
%   \;\;
%   \begin{tikzpicture}[yscale=\scaley,domain=-2.4:2.4,samples=500]
%     \draw[->] (-2.8,0) -- (2.8,0) node[above] {$f$};
%     \draw[->] (0,-0.7/\scaley) -- (0,2.4/\scaley) node[above] {$\hat{x}_p(f)$};
%     \draw[smooth,color=black,thick] plot function{\f(x-3)+\f(x-2)+\f(x-1)+\f(x)+\f(x+1)+\f(x+2)+\f(x+3)};
%     \begin{scope}[yscale=1/\scaley]
%       \vtick{1} node[pos=0.5,below] {$1$};
%       \vtick{-1} node[pos=0.5,below] {$-1$};
%       \vtick{2} node[pos=0.5,below] {$2$};
%       \vtick{-2} node[pos=0.5,below] {$-2$};
%     \end{scope}
%   \end{tikzpicture} 
% } 
% \caption{Signals $\hat{x}$ and their periodised versions $\hat{x}_p$.  Aliasing occurs in the plot on the top and middle.  No aliasing occurs in the plot on the bottom.} \label{fig:periodisedfunctions}
% \end{figure}

% \begin{excersizelist}

% \item Let $x$ be an absolutely integrable signal and let $x_p(t) = \sum_{m\in\ints} x(t - m)$ be its periodised version.  Show that $x_p$ is a periodic signal satisfying $\int_{-1/2}^{1/2} \abs{ x_p(t) } dt < \infty$.
% \begin{solution}
% We have
% \begin{align*}
% \int_{-1/2}^{1/2} \abs{ x_p(t) } dt &= \int_{-1/2}^{1/2} \abs{ \sum_{m\in\ints} x(t-m) } dt \\
% &\leq \int_{-1/2}^{1/2} \sum_{m\in\ints} \abs{  x(t-m) } dt \\
% &= \sum_{m\in\ints} \int_{-1/2}^{1/2} \abs{  x(t-m) } dt \\
% &= \sum_{m\in\ints} \int_{-1/2-m}^{1/2-m} \abs{  x(\tau) } d\tau & \text{(change variable $\tau = t - m$)}\\
% &= \int_{-\infty}^{\infty} \abs{  x(\tau) } d\tau < \infty
% \end{align*}
% because $x$ is absolutely integrable.
% \end{solution}


% \item State whether the following signals are bandlimited and, if so, find the bandwidth.
% \begin{enumerate}
% \item $\sinc(4t)$,
% \item $\rect(t/4)$,
% \item $\cos(2\pi t) \sinc(t)$,
% \item $e^{-\abs{t}}$.
% \end{enumerate}
% \begin{solution}
% Let $S_\alpha(x,t) = x(\alpha t)$ be the time scaler system.  We have 
% \begin{align*}
% \calF\big( S_\alpha(x), f\big) &= \int_{-\infty}^\infty x(\alpha t) e^{-2\pi j t } dt \\
% &= \frac{1}{\alpha} \int_{-\infty}^\infty x(\gamma) e^{ -2\pi j \gamma / \alpha  } d\gamma & \text{(ch. var. $\gamma = \alpha t$)} \\
% &= \frac{1}{\alpha} \calF\big( x, f/\alpha \big) \\
% &= \frac{1}{\alpha} S_{1/\alpha}\big(\calF(x),f\big).
% \end{align*}
% The Fourier transform of $S_\alpha(\sinc)(t) = \sinc(4t)$ is
% \[
% \calF\big( \sinc(4t) \big) = \tfrac{1}{4} \rect(f/4),
% \]
% and the signal is bandlimited with bandwidth $2$ because $\rect(f/4) = 0$ whenever $\abs{f} > 2$.  By duality
% \[
% \tfrac{1}{4} \calF(\rect(f/4)) =  \sinc(4t)
% \]
% and so $\calF(\rect(f/4)) = 4 \sinc(4t)$.  This signal is not bandlimited because the $\sinc$ function is unbounded in time.  By the modulation property of Fourier transform~\eqref{eq:modulationpropertyft},
% \[
% \calF\big( \cos(2\pi t) \sinc(t), f \big) = \calF(\sinc,f-1) + \calF(\sinc,f+1) = \rect(f-1) + \rect(f+1). 
% \]
% This is bandlimited with bandwidth $\tfrac{3}{2}$.  In Exercise~\ref{exer:fourtransealphaabs} we showed that
% \[
% \calF(e^{-\abs{t}}) =  \frac{2}{4\pi^2 f^2 + 1}.
% \]
% This signal is not bandlimited.
% \end{solution}

% \end{excersizelist}


% \subsection{The Fourier series}

% The Laplace transform and the Fourier transform do not exists for periodic signals.  In this case, the \term{Fourier series} can be used.  Let $x$ be a periodic signal with period $T$.  The $k$th \term{Fourier coefficient} of $x$ is
% \[
% \calF_s(x,k) = \int_{-T/2}^{T/2} x(t) e^{-j2\pi k t/T} dt.
% \]
% Thus, $\calF_s(x)$ is a function mapping each integer to a real or complex number, i.e., $\calF_s(x)$ is a \term{sequence}.  We write either $\calF_s(x,k)$ or $\calF_s(x)(k)$ to denote $\calF_s(x)$ evaluated at $k$.  As it was with the Fourier transform it will be convenient to let $\hat{x} = \calF_s(x)$ denote the sequence of Fourier coefficients of the periodic signal $x$.  The $k$th coefficient will be denoted by either $\hat{x}_k$ or $\hat{x}(k)$.  

% We say that the periodic signal $x$ is \term{absolutely integrable on its period} if
% \[
% \int_{-T/2}^{T/2} \abs{x(t)} dt
% \]
% is finite.  Similarly we will say that $x$ is \term{square integrable on its period} if 
% \[
% \int_{-T/2}^{T/2} \abs{x(t)}^2 dt
% \]
% is finite.  A periodic signal that is square integrable on its period must also be absolutely integrable on its period (Excersize~\ref{}).  If $x$ is absolutely integrable on it's period then its Fourier series $\hat{x} = \calF_s(x)$ is finite for all integers $k$ because
% \[
% \abs{\hat{x}(k)} = \abs{\int_{-T/2}^{T/2} x(t) e^{-j2\pi k t/T} dt} = 
% \]
% and so, 


% %\subsection{Solving differential equations with periodic inputs}
% %BLERG: Can just use the impulse response, but this might be harder that using the Fourier series.



% Let $x$ be a periodic signal with Fourier coefficients $\hat{x} = \calF_s(x)$.  The Fourier coefficients of $x^*$, the conjugate of $x$, satisfy
% \[
% \calF_s(x^*,k) = \int_{-T/2}^{T/2} \big( x(t)^* e^{-j2\pi k t/T} \big) dt.
% \]

%\section{Solving differential equations with periodic inputs}
%BLERG: Can just use the impulse response, but this might be harder that using the Fourier series.



%\chapter{Discrete-time systems}

%\section{Sampling and interpolation} \label{sec:sampl-interp}




% %\clearpage
% \section{Sampling and interpolation} \label{sec:sampl-interp}

% \section{Finite time signals}

% \section{The Nyquist theorem}

% We follow the argument given by~\cite{Shannon1949_comm_pres_noise}.  Let $x$ be a bandlimited continuous-time signal with Fourier transform $X(f) = \calF(x,f) = 0$ whenever $\abs{f} > B$.  From the inverse Fourier transform formula,
% \[
% x(t) = \frac{1}{2\pi}\int_{-\infty}^\infty X(f) e^{j2\pi ft} dt = \frac{1}{2\pi}\int_{-B}^{B} X(f) e^{j2\pi ft} dt.
% \]
% Sampling $x(t)$ with period $T_s = \frac{1}{f_s}$ we obtain
% \[
% y(n) = x(nT) = \frac{1}{2\pi}\int_{-B}^{B} X(f) e^{j2\pi f n T} df
% \]
% and provided that $f_s \geq 2B$
% \[
% y(n) = x(nT) = \frac{1}{2\pi}\int_{-f_s/2}^{f_s/2} X(f) e^{-j2\pi f n / f_s} dt
% \]
% Let $\tilde{x}(f)$ be the signal with period $f_s$ defined by 
% \[
% \tilde{x}(f) = \sum_{n\in\ints} X(f + nf_s)
% \] 
% and observe that $X$ can be recovered from $\tilde{x}$ by, for example, multiplying by the rectangle function $\Pi(t f_s)$, that is $X(f) = Pi(f f_s) \tilde{x}(f)$.  In this way $\tilde{x}$ uniquely determines $X$.  Now
% \[
% y(n) = x(nT) = \frac{1}{2\pi}\int_{-f_s/2}^{f_s/2} \tilde{x}(-t) e^{-j2\pi t n / f_s} dt
% \]
% after a change of variable $t = -f$.  Observe that $y(n)$ is the $n$ Fourier coefficient of the periodic signal $\tilde{x}(-t)$, and so $\tilde{x}$ is uniquely determined by the discrete-time signal $y$.  Since the Fourier transform $X(f)$ is uniquely determined by $\tilde{x}$ and since $X(f)$ uniquely determines the original continuous-time signal $x$, we have that the discrete-time signal $y$ uniquely determines $x$.


% \section{Discrete-time signals}


% \section{Exercises}

% \begin{excersizelist}

% \item \label{excer:disctimeeneryexpbound} Show that the discrete-time signal $e^{-n^2/4}$ is an energy signal and that its energy is less than $\sqrt{2\pi} + 1$.  (Hint: Exercise~\ref{excer:energyexpchangevar} first) 

% \item \label{excer:disctimeabssummableexpbound} Show that  the discrete-time signal $e^{-n^2/4}$ is absolutely summable. (Hint: do Exercise~\ref{excer:disctimeeneryexpbound} first).

% \end{excersizelist}


% \section{The Z-transform}

% \section{Digital filters}


% %\clearpage

\backmatter

 %\appendix

% \chapter{Estimating the output and input resistance of your soundcard}\label{sec:estim-outp-input}

% \begin{figure}[tp]
% \centering
% \begin{circuitikz} \draw
%  %(0,0) node[anchor=east]{B}
%   to[R,l=$R_o$] (3,0)
%   %to[R,l=$R_\ell$] (3,3)
%  (3,3) to[R, l=$R_\ell$, -o] (0,3)
%  to[open, v=$x(t)$] (0,0)
%  (3,0) to[short,-o] (4,0)
%  (3,3) to[short,-o] (4,3)
%  (3,0) to[R,l_=$R_i$] (3,3)
%  (4,0) to[open, v>=$y(t)$] (4,3)
% ;\end{circuitikz}
% \caption{A \term{voltage divider} circuit including input resistance $R_i$ and output resistance $R_o$.} \label{circ:voltagedividerinpouttoestimate}
% \end{figure}

% Construct the circuit~\ref{circ:voltagedividerinpouttoestimate} by placing a single resistor $R_\ell$ in parallel between the input and output device of the soundcard.  We will use $K$ different values of $R_\ell$. The value of the resistor is free to be choosen.  An approximation of the signal
% \[
% x(t) = \sin\big( 2\pi (\alpha t + \beta t^2) \big), \qquad \alpha = 100, \beta = 7450
% \]
% is passed through the circuit.  This type of signal is called a \term{quadratic chirp}.  The idea being that (approximately) frequencies from range $\alpha = 100\si{\hertz}$ to $\alpha + t_{\max}\beta = 15\kilo\si{\hertz}$ are represented by the signal, where $t_{max}=2$ is the duration of the signal (two seconds here).  The approximation of $x$ is generated by sampling $x(t)$ at rate $F_s = \frac{1}{T_s} = 44100\si{\hertz}$ to generate samples 
% \[
% x_{\ell, n} = x(n T_s) \qquad n = 0, \dots, 2 F_s.
% \]
% corresponding to approximately $t_{max} = 2$ seconds of signal.  These samples are passed to the soundcard which starts playback.  The voltage over the resistor $R_{\ell n}$ is recorded (also using the soundcard) that returns a lists of samples $y_{\ell 1},\dots,y_{\ell L}$.  Simultaneously the (stereo) soundcard is used to record the voltage over the output of the soundcard into samples $x_{\ell 1}, \dots, x_{\ell L}$.

% A subset of the recorded samples
% \[
% x_{\ell s}, \dots, x_{\ell e}, \qquad y_{\ell s}, \dots, y_{\ell e}
% \]
% where $s = \floor{F_s/2}$ and $e = \floor{3F_s/2}$ is taken.  These samples for $\ell = 1, \dots, K$ will be used to estimate the input and output resistors $R_o$ and $R_i$.  The circuit acts as a voltage divider and the relationship between $x$ and $y$ is
% \[
% y = \frac{R_i}{R_i + R_o + R_\ell} x = A_\ell x
% \]
% We will makes least squares estimates of $A_1$ and $A_2$.  They are the minimisers of
% \[
% S_{\ell}(a) = \sum_{n = s}^{e} \big(y_{\ell n} - a x_{\ell n}\big)^2
% \]
% Differentiating with respect to $a$, and setting to zero
% \[
% 0 = - 2 \sum_{n = s}^{e} x_{\ell n} \big(y_{\ell n} - \hat{a}_\ell x_{\ell n}\big)
% \]
% from which we obtain the mnimiser
% \[
% \hat{a}_\ell = \frac{\sum_{n = s}^{e} x_{\ell n} y_{\ell n}}{\sum_{n = s}^{e} x_{\ell n} x_{\ell n}}.
% \]
% Given these too estimates $\hat{a}_1$ and $\hat{a}_2$ we obtain estimates of $R_o$ and $R_i$ by solving the simultaneous equations
% \[
% \hat{a}_1 = \frac{\hat{R}_i}{\hat{R}_0 + R_1 + \hat{R}_i}, \qquad \hat{a}_2 = \frac{\hat{R}_i}{\hat{R}_0 + R_2 + \hat{R}_i}.
% \]
% Solving these equations
% \[
% \hat{R}_i = \frac{\hat{a}_1\hat{a}_2(R_2 - R_1)}{\hat{a}_1 - \hat{a}_2}, \qquad \hat{R}_0 = \frac{1-\hat{a}_1}{\hat{a}_1}R_i - R1.
% %\hat{R}_0 = \frac{\hat{a}_1(\hat{a}_2-1)R_1 + \hat{a}_2(1 - \hat{a}_1)R_2}{\hat{a}_1 - \hat{a}_2}
% \]

% \section{Least squares estimation of time offset and amplitude}\label{app:least-squar-estim}

% During most of the tests in this course we have used a synchronisation circuit to estimate the gain and time delay caused by the computers soundcard.  A signal $x$ is output by the soundcard and sampled to obtain
% \[
% s_n =  \alpha_0 x(n T_s - \tau_0) + \epsilon_n, \qquad n = 1, \dots, L
% \]
% where $\alpha_0$ and $\tau_0$ where $\tau_0$ represents the unknown time shift and $\alpha_0$ the unknown gain and $\epsilon_1,\dots,\epsilon_L$ represent noise.  We will estimate $\alpha_0$ and $\tau_0$ using a least squares method.  That is, we choose estimates $\hat{\alpha}$ and $\hat{\tau}$ to be the minimisers of the sum of squares function
% \[
% S(\alpha, \tau) = \sum_{n=1}^L \big( s_n - \alpha x(n T_s - \tau) \big)^2.
% \]
% For fixed $\tau$ the minimiser of $S$ with respect to $\alpha$ is
% \begin{align*}
% \alpha(\tau) &= \arg\min_{\alpha \in \reals} S(\alpha,\tau) \\
% &= \frac{\sum_{n=1}^L s_n x(n T_s - \tau)}{\sum_{n=1}^L \big(x(n T_s - \tau)\big)^2 },
% \end{align*}
% which is readily shown by differentiating $S$ with respect to $\alpha$ and setting the result to zero.  Substituting this into $S(\alpha, \tau)$ we obtain $S$ minimised with respect to $\alpha$,
% \[
% S(\tau) = \min_{\alpha\in\reals} S(\tau,alpha) = \sum_{n=1}^L \big( s_n - \alpha(\tau) x(n T_s - \tau) \big)^2.
% \]
% that is only a function of $\tau$.  We now assume that the true time offset $\tau_0$ in known to lie in some interval $[\tau_{\text{\min}}, \tau_{\text{\max}}]$, so that we want to compute
% \[
% S(
% \]


%%% Local Variables: 
%%% mode: latex
%%% TeX-master: "main.tex"
%%% End: 
