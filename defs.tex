% \newcommand {\tit}[1] {\textit{#1}}
% \newcommand {\tmd}[1] {\textmd{#1}}
% \newcommand {\trm}[1] {\textrm{#1}}
% \newcommand {\tsc}[1] {\textsc{#1}}
% \newcommand {\tsf}[1] {\textsf{#1}}
% \newcommand {\tsl}[1] {\textsl{#1}}
% \newcommand {\ttt}[1] {\texttt{#1}}
% \newcommand {\tup}[1] {\textup{#1}}

% \newcommand {\mmd}[1] {\mathmd{#1}}
% \newcommand {\mrm}[1] {\mathrm{#1}}
% \newcommand {\msc}[1] {\mathsc{#1}}
% \newcommand {\msf}[1] {\mathsf{#1}}
% \newcommand {\msl}[1] {\mathsl{#1}}
% \newcommand {\mtt}[1] {\mathtt{#1}}
% \newcommand {\mup}[1] {\mathup{#1}}

%some math functions and symbols
\newcommand{\round}[1]{\left\lceil #1 \right\rfloor}
\newcommand{\floor}[1]{\left\lfloor #1 \right\rfloor}
\newcommand{\ceil}[1]{\left\lceil #1 \right\rceil}
\newcommand{\reals}{{\mathbb R}}
\newcommand{\ints}{{\mathbb Z}}
\newcommand{\complex}{{\mathbb C}}
\newcommand{\integers}{{\mathbb Z}}
\newcommand{\sign}{\mathtt{sign}}
\newcommand{\NP}{\operatorname{NearestPt}}
\newcommand{\NS}{\operatorname{NearestSet}}
\newcommand{\bres}{\operatorname{Bres}}
\newcommand{\vol}{\operatorname{vol}}
\newcommand{\vor}{\operatorname{Vor}}
\newcommand{\coef}{\operatorname{coef}}
\newcommand{\eval}{\operatorname{eval}}
\newcommand{\Int}{\operatorname{Int}}
\newcommand{\pvec}{\operatorname{vec}}
\newcommand{\rem}{\operatorname{rem}}
\newcommand{\var}{\operatorname{var}}
\newcommand{\covar}{\operatorname{covar}}
\newcommand{\erf}{\operatorname{erf}}
\newcommand{\adj}{\operatorname{adj}}
\newcommand{\pad}{\operatorname{pad}}
\newcommand{\dealias}{\operatorname{dealias}}
\newcommand{\prob}{\operatorname{Pr}}

%distribution fucntions
\newcommand{\projnorm}{\operatorname{ProjectedNormal}}
\newcommand{\vonmises}{\operatorname{VonMises}}
\newcommand{\wrapnorm}{\operatorname{WrappedNormal}}
\newcommand{\wrapunif}{\operatorname{WrappedUniform}}

%sorting and selecting
\newcommand{\selectindicies}{\operatorname{selectindices}}
\newcommand{\sortindicies}{\operatorname{sortindices}}
\newcommand{\largest}{\operatorname{largest}}
\newcommand{\quickpartition}{\operatorname{quickpartition}}
\newcommand{\quickpartitiontwo}{\operatorname{quickpartition2}}

%caligraphic letters, b means bold.
\newcommand{\bcalL}{\bm{\mathcal{L}}}
\newcommand{\bcalX}{\bm{\mathcal{X}}}
\newcommand{\bcalP}{\bm{\mathcal{P}}}
\newcommand{\calP}{\mathcal{P}}
\newcommand{\calR}{\mathcal{R}}
\newcommand{\calV}{\mathcal{V}}


%Brackets
\newcommand{\br}[1]{{\left( #1 \right)}}
\newcommand{\sqbr}[1]{{\left[ #1 \right]}}
\newcommand{\cubr}[1]{{\left\{ #1 \right\}}}
\newcommand{\abr}[1]{\left< #1 \right>}
\newcommand{\abs}[1]{{\left| #1 \right|}}
\newcommand{\ceiling}[1]{{\left\lceil #1 \right\rceil}}
\newcommand{\magn}[1]{\left\| #1 \right\|}
\newcommand{\fracpart}[1]{\left< #1 \right>}
\newcommand{\dotprod}[2]{ #1 \cdot #2}

%some commonly used underlined and
%hated symbols
\newcommand{\uY}{\ushort{\mbf{Y}}}
\newcommand{\ueY}{\ushort{Y}}
\newcommand{\uy}{\ushort{\mbf{y}}}
\newcommand{\uey}{\ushort{y}}
\newcommand{\ux}{\ushort{\mbf{x}}}
\newcommand{\uex}{\ushort{x}}
\newcommand{\uhx}{\ushort{\mbf{\hat{x}}}}
\newcommand{\uehx}{\ushort{\hat{x}}}

\newcommand {\figwidth} {100mm}
\newcommand {\Ref}[1] {Reference~\cite{#1}}
\newcommand {\Sec}[1] {Section~\ref{#1}}
\newcommand {\App}[1] {Appendix~\ref{#1}}
\newcommand {\Chap}[1] {Chapter~\ref{#1}}
\newcommand {\etal} {\emph{~et~al.}}
\newcommand {\bul} {$\bullet$ }   % bullet
\newcommand {\fig}[1] {Figure~\ref{#1}}   % references Figure x
\newcommand {\imp} {$\Rightarrow$}   % implication symbol (default)
\newcommand {\impt} {$\Rightarrow$}   % implication symbol (text mode)
\newcommand {\impm} {\Rightarrow}   % implication symbol (math mode)
\newcommand {\vect}[1] {\mathbf{#1}} 
\newcommand {\hvect}[1] {\hat{\mathbf{#1}}}
\newcommand {\del} {\partial}
\newcommand {\eqn}[1] {Equation~(\ref{#1})} 
\newcommand {\tab}[1] {Table~\ref{#1}} % references Table x
\newcommand {\half} {\frac{1}{2}} 
\newcommand {\ten}[1] {\times10^{#1}}
\newcommand {\bra}[2] {\mbox{}_{#2}\langle #1 |}
\newcommand {\ket}[2] {| #1 \rangle_{#2}}
\newcommand {\Bra}[2] {\mbox{}_{#2}\left.\left\langle #1 \right.\right|}
\newcommand {\Ket}[2] {\left.\left| #1 \right.\right\rangle_{#2}}
\newcommand {\im} {\mathrm{Im}}
\newcommand {\re} {\mathrm{Re}}
\newcommand {\braket}[4] {\mbox{}_{#3}\langle #1 | #2 \rangle_{#4}} 
%\newcommand {\dotprod}[4] {\mbox{}_{#3}\langle #1 | #2 \rangle_{#4}} 
\newcommand {\trace}[1] {\text{tr}\left(#1\right)}

% spell things correctly
\newenvironment{centre}{\begin{center}}{\end{center}}
\newenvironment{itemise}{\begin{itemize}}{\end{itemize}}

%algorithm writing package
\usepackage[vlined, linesnumbered, algochapter]{algorithm2e}
%\usepackage[vlined, linesnumbered, algochapter, oldcommands]{algorithm2e}
\SetAlCapFnt{\small}
\SetAlTitleFnt{textsc}

%\usepackage{play}
%\usepackage[grey,times]{quotchap}
\usepackage[times]{quotchap}
%\usepackage{makeidx}

%%%%% set up the bibliography style
\bibliographystyle{plainnat}
%\bibliographystyle{robbythesis}
%\bibliographystyle{anuthesis}
%\bibliographystyle{uqthesis}  % uqthesis bibliography style file, made
			      % with makebst

%%%%% optional packages
\usepackage[square]{natbib}
		% this is the natural sciences bibliography citation
		% style package.  The options here give citations in
		% the text as numbers in square brackets, separated by
		% commas, citations sorted and consecutive citations
		% compressed 
		% output example: [1,4,12-15]
		
%\usepackage{multibib}
%\newcites{anstar}{ANstat}
%\newcites{sec}{List of Publications}
%\usepackage[subsectionbib]{bibunits}
	%multiple bibliographies, ie a list of publications
	
\usepackage{units}
	%nice looking units
		
\usepackage{booktabs}
		%creates nice looking tables

%glossay package		
%\usepackage{glossary}
%\newcommand{\glosterm}[1]{\textbf{#1}}
%\newcommand{\defglosterm}[2]{\glosterm{#1}\glossary{name={#1}, description={#2}}}
%\makeglossary

\usepackage{makeidx}
\newcommand{\keywordfront}[1]{\textbf{#1}}
%notation use when defining terms.
\newcommand{\term}[1]{\keywordfront{#1}}
\makeindex
		
  \usepackage[pdftex]{graphicx}
%  \usepackage{thumbpdf}
  %\usepackage[naturalnames]{hyperref}
%  \usepackage{hyperref}
 % \DeclareGraphicsRule{*}{mps}{*}{}

\usepackage{amsmath,amsfonts,amssymb, amsthm} % this is handy for mathematicians and physicists
			      % see http://www.ams.org/tex/amslatex.html

%\usepackage[intoc]{nomencl}
%\usepackage{showkeys} % this shows what labels you are using for cross
		      % references
		      
	
\usepackage{mathrsfs}
%fancy math script

\usepackage{ushort}
%enable good underlining in math mode

%subfigures
\usepackage{subfigure}

\usepackage[nottoc]{tocbibind}  
				% allows the table of contents, bibliography
				% and index to be added to the table of
				% contents if desired, the option used
				% here specifies that the table of
				% contents is not to be added.
				% tocbibind needs to be after natbib
				% otherwise bits of it get trampled.
				
		 
	
%------------------------------------------------------------
% Theorem like environments
%
\newtheorem{theorem}{Theorem}[chapter]
\theoremstyle{plain}
\newtheorem{acknowledgement}{Acknowledgement}[chapter]
%\newtheorem{algorithm}{Algorithm}
\newtheorem{axiom}{Axiom}[chapter]
\newtheorem{case}{Case}[chapter]
\newtheorem{claim}{Claim}[chapter]
\newtheorem{conclusion}{Conclusion}[chapter]
\newtheorem{condition}{Condition}[chapter]
\newtheorem{conjecture}{Conjecture}[chapter]
\newtheorem{corollary}{Corollary}[chapter]
\newtheorem{criterion}{Criterion}[chapter]
\newtheorem{definition}{Definition}[chapter]
\newtheorem{example}{Example}[chapter]
\newtheorem{exercise}{Exercise}[chapter]
\newtheorem{lemma}{Lemma}[chapter]
\newtheorem{notation}{Notation}[chapter]
\newtheorem{problem}{Problem}[chapter]
\newtheorem{proposition}{Proposition}[chapter]
\newtheorem{remark}{Remark}[chapter]
\newtheorem{solution}{Solution}[chapter]
\newtheorem{summary}{Summary}[chapter]
\newtheorem{result}{Result}[chapter]
\numberwithin{equation}{section}
%--------------------------------------------------------

%%% Local Variables: 
%%% mode: latex
%%% TeX-master: "writen_component"
%%% End: 
