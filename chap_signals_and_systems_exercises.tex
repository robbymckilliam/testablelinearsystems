\section{Exercises}

\begin{excersizelist}

\item How many distinct functions from the set $X = \{\text{Mario}, \text{Link}\}$ to the set $Y = \{\text{Freeman}, \text{Ryu}, \text{Sephiroth}\}$ exist?  Write down each function from $X \to Y$.

\begin{solution}
Each of the two elements in $X$ can be mapped to the three elements of $Y$.  There are thus $3^2 = 9$ distinct functions in $X \to Y$.  They are
\[
f_1(x) = \begin{cases}
\text{Freeman} & x = \text{Mario} \\
\text{Freeman} & x = \text{Link} \\
\end{cases}
\qquad 
f_2(x) = \begin{cases}
\text{Freeman} & x = \text{Mario} \\
\text{Ryu} & x = \text{Link} \\
\end{cases}
\]
\[
f_3(x) = \begin{cases}
\text{Ryu} & x = \text{Mario} \\
\text{Freeman} & x = \text{Link} \\
\end{cases}
\qquad 
f_4(x) = \begin{cases}
\text{Freeman} & x = \text{Mario} \\
\text{Sephiroth} & x = \text{Link} \\
\end{cases}
\]
\[
f_5(x) = \begin{cases}
\text{Sephiroth} & x = \text{Mario} \\
\text{Freeman} & x = \text{Link} \\
\end{cases}
\qquad 
f_6(x) = \begin{cases}
\text{Ryu} & x = \text{Mario} \\
\text{Ryu} & x = \text{Link} \\
\end{cases}
\]
\[
f_7(x) = \begin{cases}
\text{Ryu} & x = \text{Mario} \\
\text{Sephiroth} & x = \text{Link} \\
\end{cases}
\qquad 
f_8(x) = \begin{cases}
\text{Sephiroth} & x = \text{Mario} \\
\text{Ryu} & x = \text{Link} \\
\end{cases}
\]
\[
f_9(x) = \begin{cases}
\text{Sephiroth} & x = \text{Mario} \\
\text{Sephiroth} & x = \text{Link} \\
\end{cases}
\]
\end{solution}

\item \label{excer:stepfunction} State whether the step function $u(t)$ is bounded, periodic, %continuous, 
absolutely integrable, an energy signal.
\begin{solution}
The magnitude of $u$ is less than or equal to one, so the signal is bounded.  The signal is not period, since for any hypothesised period $T > 0$ we have $u(T) = 1$ but $u(0) = 0$.  %The signal is not continuous at $t = 0$ since $\lim_{t \to 0^+} u(t) = 0$ and $\lim_{t \to 0^{-}} u(t) = 1$.  
The signal is not absolutely integrable, nor an energy signal since
\[
\|u\|_1 = \|u\|_2 = \int_{-\infty}^\infty \abs{u(t)} dt = \int_{0}^\infty dt
\]
is not finite.
\end{solution}

\item \label{exer:oneontnotlocallyint} Show that the signal $t^2$ is locally integrable, but that the signal $\frac{1}{t^2}$ is not. 

\begin{solution}
For any $a$ and $b$
\[
\int_a^b t^2 dt = \frac{b^3}{3} - \frac{a^3}{3}
\]
is finite and so $t^2$ is locally integrable.  Put $a = 0$ and $b > 0$ and
\[
\int_0^b \frac{1}{t^2} dt = -\frac{1}{b} + \lim_{t \to 0}\frac{1}{t}  = \infty.
\]
The limit above diverges and so $\frac{1}{t^2}$ is not locally integrable. 
\end{solution}

% \item \label{exer:boundedfunctionarelocallyintegrable} Show that every bounded function is locally integrable.  

\item \label{exer:functionsquarenotabsint} Plot the signal 
\[
x(t) = \begin{cases}
\tfrac{1}{t+1} & t > 0 \\
\tfrac{1}{t-1} & t \leq 0.
\end{cases}
\]
State whether it is: bounded, locally integrable, absolutely integrable, square integrable.

\begin{solution}
\begin{center}
  \begin{tikzpicture}[samples=200]
    % \draw[very thin,color=gray] (-0.1,-1.1) grid (3.9,3.9);
    \draw[->] (-4.5,0) -- (4.5,0) node[above] {$t$};
    \draw[->] (0,-1.2) -- (0,1.2);
    \draw[smooth,color=black,thick,domain=0:4] plot function{1/(x+1)};
    \draw[smooth,color=black,thick,domain=-4:0] plot function{1/(x-1)};
    \htick{1} node[pos=0.5,above right] {$1$};
    \vtick{-4} node[pos=0.5,above] {$-4$};
    \vtick{-2} node[pos=0.5,above] {$-2$};
    \vtick{2} node[pos=0.5,below] {$2$};
    \vtick{4} node[pos=0.5,below] {$4$};
  \end{tikzpicture}
\end{center}
The signal is bounded since $\abs{x(t)} < M$ for any $M > 1$.  The signal is locally integrable because it is bounded, i.e., for any finite constants $a$ and $b$
\[
\int_{a}^b \abs{x(t)} dt < \int_{a}^b M dt = (b-a)M < \infty.
\]
The signal $x$ is not absolutely integrable since
\begin{align*}
\|x\|_1 &= \int_{-\infty}^\infty \abs{x(t)} dt \\
&= 2 \int_{0}^\infty \frac{1}{t+1} dt \\
&= 2 \int_{1}^\infty \frac{1}{t} dt \\
&= 2\log(1) + \lim_{t \to \infty} 2\log(t)
\end{align*}
and the limit diverges.  The signal is square integrable since
\begin{align*}
\|x\|_2 &= \int_{-\infty}^\infty \sabs{x(t)}^2 dt \\
&= 2 \int_{0}^\infty \frac{1}{(t+1)^2} dt \\
&= 2 \int_{1}^\infty \frac{1}{t^2} dt \\
&= 2 - \lim_{t \to \infty} \frac{2}{t} = 2.
\end{align*}
\end{solution}

\item \label{excer:absintnotsquareint} Plot the signal
\[
x(t) = \begin{cases}
\frac{1}{\sqrt{t}} & 0 < t \leq 1 \\
0 & \text{otherwise}.
\end{cases}
\]
Show that $x$ is absolutely integrable, but not square integrable.

\begin{solution}
\begin{center}
  \begin{tikzpicture}[samples=50]
    % \draw[very thin,color=gray] (-0.1,-1.1) grid (3.9,3.9);
    \draw[->] (-4.5,0) -- (4.5,0) node[above] {$t$};
    \draw[->] (0,-0.2) -- (0,2.5);
    \draw[thick] (-4,0)--(0,0);
    \draw[thick,line cap=round] (1,1)--(1,0)--(4,0);
    \draw[smooth,color=black,thick,domain=0.2:1] plot function{1.0/sqrt(x)};
    \htick{1} node[pos=0.5,left] {$1$};
    \vtick{-3} node[pos=0.5,below] {$-3$};
    \vtick{-1} node[pos=0.5,below] {$-1$};
    \vtick{1} node[pos=0.5,below] {$1$};
    \vtick{3} node[pos=0.5,below] {$3$};
  \end{tikzpicture}
\end{center}
The integral
\[
\|x\|_1 = \int_{-\infty}^\infty \abs{x(t)} dt = \int_{0}^1 t^{-1/2} dt = [2\sqrt{t}]_0^1 = 2
\]
and so $x$ is absolutely integrable.  The integral
\[
\|x\|_2 = \int_{-\infty}^\infty \abs{x(t)} dt = \int_{0}^1 t^{-1} dt = [\log(t)]_0^1 = \log(1) - \lim_{t \to 0}\log(t) = \infty
\]
and so $x$ is not square integrable.
\end{solution}

\item \label{excer:energyexpchangevar} Compute the energy of the signal $e^{-\alpha^2 t^2}$ (Hint: use equation~\eqref{eq:expabssum} on page~\pageref{eq:expabssum} and a change of variables).
\begin{solution}
From~\eqref{eq:expabssum} we the energy of $e^{-t^2}$ is $\sqrt{\pi}$.  Now
\[
\int_{-\infty}^\infty e^{-\alpha t^2} dt = \frac{1}{\alpha}\int_{-\infty}^\infty e^{-\tau^2} d\tau = \frac{\sqrt{\pi}}{\alpha}
\]
by the change of variables $\tau = \alpha t$.
\end{solution}

\item \label{exer:steprectnotdiff}Show that the signal $t^2$ is differentiable, but the step function $u$ and rectangular pulse $\rect$ are not.
\begin{solution}
We have
\[
\lim_{h \to 0} \frac{(t + h)^2 - t^2}{h} = \lim_{h \to 0} \frac{2 t h + h^2}{h} = 2t.
\]
\[
\lim_{h \to 0} \frac{t^2 - (t-h)^2}{h} = \lim_{h \to 0} \frac{2 t h - h^2}{h} = 2t
\]
and so $t^2$ is continuously differentiable with derivative $\frac{d}{dt} t^2 = 2t$.  At $t = 0$ the corresponding limits for the step function are
\[
\lim_{h \to 0} \frac{u(h) - u(0)}{h} = \lim_{h \to 0} \frac{0}{h} = 0
\]
but
\[
\lim_{h \to 0} \frac{u(0) - u(-h)}{h} = \lim_{h \to 0} \frac{1}{h} = \infty
\]
so the step function $u$ is not differentiable at $t = 0$.  A similar argument at $t=\tfrac{1}{2}$ or $t = -\tfrac{1}{2}$ shows that $\rect$ is not differentiable.
\end{solution}

\item \label{exer:L1L2linearshiftinvariant} Show that the set of absolutely integrable signals $L_1$ and the set of square integrable signals $L_2$ are linear space and that they are shift-invariant. 

\item \label{exer:periodicshiftinvariantnotlinear} Show that the set of periodic signals is shift-invariant, but is not a linear space.

\item \label{exer:boundedlinearshiftinvar} Show that the set of bounded signals is shift-invariant and is a linear space.

\item \label{exer:boundconstantnotlinear} Let $M > 0$ be a fixed real number. Show that the set of signals bounded below $M$ is shift invariant, but is not a linear space. 

\item \label{exer:boundedlinearshiftinvar} Show that the set of right sided signals is shift-invariant and is a linear space.

\item \label{exer:evenoddnoshiftinvariant} Show that the set of even signals and the set of odd signals are not shift invariant.

\item \label{excer:integratornotstable} Show that the integrator $I_a$ for any $a\in\reals$ is not  stable.
\begin{solution}
Put $M > 1$.  The time-shifted step function $u(t + a)$ is bounded below $M$, i.e. $\sabs{u(t+a)} \leq 1 < M$ for all $t \in \reals$.  However, the response of the integrator $I_a$ to $u(t+a)$ is
\[
I_a\big(u(t+a)\big) = \int_{-a}^t u(\tau + a)d\tau = \begin{cases}
\int_{-a}^t d\tau = t + a & t \geq -a \\
0 & t < -a 
\end{cases},
\]
and this is not a bounded signal, that is, for every $K$ we have $t + a > K$ whenever $t > K - a$.
\end{solution}

\item \label{excer:diffnotstable} Show that the differentiator system $D$ is not  stable.
\begin{solution}
Put $M > 2$.  Define the signal
\[
q_a(t) = \begin{cases}
0 & 2t < -a \\
1 + \sin\big(\tfrac{\pi t}{a}\big) & -a < 2t < a \\
2 & 2t > a,
\end{cases}
\]
and observe that $q_a$ is differentiable and bounded below $M$.  The response of the differentiator $D$ to $q_a$ is
\[
D(q_a,t) = \begin{cases}
0 & 2t < -a \\
\tfrac{\pi}{a} \cos\big(\tfrac{\pi t}{a}\big) & -a < 2t < a \\
1 & 2t > a.
\end{cases}
\]
The signal $p_a$ and the response $D(p_a)$ are plotted below for $a = \tfrac{1}{2},1$ and $2$.  The response $D(p_a)$ obtains a maximum amplitude of $\tfrac{\pi}{a}$ at $t=0$.  So $D$ is not stable because for any $K$ we can choose $a < \tfrac{\pi}{K}$ so that $\tfrac{\pi}{a} > K$.

\newcommand{\sinpulse}[1]{
\draw[color=black,thick] (-1.5,0) -- (-#1/2,0) node {};
\draw[smooth,color=black,thick,domain=-#1/2.0:#1/2.0] plot function{1 + sin(3.14159265359*x/#1)};
\draw[color=black,thick] (#1/2,2) -- (1.5,2)  node {};
}
\newcommand{\responsesinpulse}[1]{
\draw[color=black,thick] (-1.5,0) -- (-#1/2,0) node {};
\draw[smooth,color=black,thick,domain=-#1/2.0:#1/2.0] plot function{3.14159265359/#1*cos(3.14159265359*x/#1)};
\draw[color=black,thick] (#1/2,0) -- (1.5,0)  node {};
}
\newcommand{\sinpulseresponse}[1]{}
\begin{center}
  \begin{tikzpicture}
    % \draw[very thin,color=gray] (-0.1,-1.1) grid (3.9,3.9);
    \draw[->] (-2,0) -- (2,0) node[above] {$t$};
    \draw[->] (0,-0.5) -- (0,2.5) node[right] {$q_a(t)$};
    \sinpulse{1};
    \sinpulse{0.5};
    \sinpulse{2};
  \end{tikzpicture}
\;\;
  \begin{tikzpicture}
    \begin{scope}[yscale=0.5]
    % \draw[very thin,color=gray] (-0.1,-1.1) grid (3.9,3.9);
    \draw[->] (-2,0) -- (2,0) node[above] {$t$};
    \draw[->] (0,-1) -- (0,7) node[right] {$q_a(t)$};
    \responsesinpulse{1};
    \responsesinpulse{0.5};
    \responsesinpulse{2};
    \end{scope}
  \end{tikzpicture}
\end{center}
\end{solution}

\item Show that the time-shifter is linear and time-invariant and that the time-scaler is linear, but not time invariant
\begin{solution}
The time shifter $T_\tau$ is time invariant since
\[
T_k\big(T_\tau(x)\big) = T_k\big(x(t - \tau)\big) = x(t - \tau - k) = x(t - k - \tau) = T_\tau\big(x(t - k)\big) = T_\tau\big(T_k(x)\big) 
\]
for all signals $x$, that is, time shifters commute with time shifters.  The time shifter is linear because
\[
T_\tau(ax + by) = ax(t - \tau) + by(t - \tau) = a T_\tau(x) + b T_\tau(y).
\]
The time-scaler $H(x) = x(\alpha t)$ is linear because
\[
H(ax + by) = ax(\alpha t) + by(\alpha t) = aH(x) + b H(y).
\]
The system is not time invariant because
\[
H\big(T_\tau(x)\big) = H\big(x(t-\tau)\big) = x(\alpha t - \tau)
\]
but 
\[
T_\tau\big(H(x)\big) = T_\tau\big(x(\alpha t) \big) = x(\alpha(t - \tau)) = x( \alpha t - \alpha \tau ),
\]
and these signals are not equal in general.  For example consider the rectangular pulse $\Pi$.  With time scaling parameter $\alpha = 2$ and time shift $\tau = 1$,
\[
H\big(T_1(\Pi)\big) = \Pi( 2 t - 1 ) \neq \Pi( 2t - 2 ) = T_1\big(H(\Pi)\big).
\]
\end{solution}

%\item \label{excer:difflLTI} Show that the $k$th differentiator $D^k(x,t) = \tfrac{d^k}{dt^k} x(t)$ is linear and time-invariant

\item Show that the integrator $I_c$ with $c$ finite is linear, but not time-invariant.
\begin{solution}
The system is linear because
\begin{align*}
I_c(ax + by) &= \int_{-c}^t ax(\tau) + b y(\tau) d\tau \\
&= a\int_{-c}^t x(\tau) d\tau + b \int_{-c}^t y(\tau) d\tau \\
&= a I_c(x)  + b I_c(y).
\end{align*}
The system is not time invariant because
\[
T_k\big(I_c(x)\big) = I_c(x,t-k) = \int_{-c}^{t-k} x(\tau) d\tau 
\]
but
\[
I_c\big(T_k(x)\big) = \int_{-c}^{t} x(\tau-k) d\tau.
\]
We now need only find some signal $x$ for which the integrals on the right hand side of the above equations are not equal.  Choose the signal $x = 1$, i.e., the signal that is equal to $1$ for all time.  In this case
\[
T_k\big(I_c(1)\big) = \int_{-c}^{t-k} d\tau =  t-k+c \neq t + c = \int_{-c-k}^{t-k} d\tau = I_c\big(T_k(1)\big) \qquad \text{when $k \neq 0$.}
\]
\end{solution}

\item Show that the integrator $I_\infty$ is linear and time invariant.
\begin{solution}
The system is linear because
\begin{align*}
I_\infty(ax + by) &= \int_{-\infty}^t ax(\tau) + b y(\tau) d\tau \\
&= a\int_{-\infty}^t x(\tau) d\tau + b \int_{-\infty}^t y(\tau) d\tau \\
&= a I_\infty(x)  + b I_\infty(y).
\end{align*}
The system is time invariant because
\[
T_k\big(I_\infty(x)\big) = I_\infty(x,t-k) = \int_{-\infty}^{t-k} x(\tau) d\tau,
\]
and
\[
I_\infty\big(T_k(x)\big) = \int_{-\infty}^{t} x(\tau-k) d\tau = \int_{-\infty}^{t-k} x(\tau) d\tau.
\]
\end{solution}

\item State whether the system $H(x) = x + 1$ is linear, time-invariant,  stable.
\begin{solution}
It is not linear because for any signal $x$ and real number $a \neq 1$,
\[
H(ax) = ax + 1 \neq aH(x) = a\big( x + 1\big) = ax + a.
\]
It is time-invariant because
\[
H\big(T_{\tau}(x)\big) = x(t - \tau) + 1 = T_\tau(x + 1) = T_\tau\big(H(x)\big).
\]
It is  stable because for any signal $x$ with $x(t) < M$ for all $t\in\reals$, 
\[
H(x,t) = x(t) + 1 < M+1 \qquad \text{for all $t \in \reals$}. 
\]
\end{solution}

\item State whether the system $H(x) = 0$ is linear, time-invariant,  stable.
\begin{solution}
It is linear because
\[
H(ax + by) = 0 = aH(x) + bH(y) = 0.
\]
It is time-invariant because
\[
H\big(T_{\tau}(x)\big)\big(t\big) = 0 = H(x,t-\tau).
\]
It is stable because for any $K > 0$,
\[
H(x,t) = 0 < K \qquad \text{for all $t \in \reals$ and all signals $x$}. 
\]
\end{solution}

\item State whether the system $H(x) = 1$ is linear, time-invariant,  stable.
\begin{solution}
It is not linear because for any signal $x$ and real number $a \neq 1$
\[
H(ax) = 1 \neq  aH(x) = a.
\]
It is time-invariant because
\[
H\big(T_{\tau}(x)\big) = 1 = T_\tau(1) = T_\tau\big(H(x)\big).
\]
It is  stable because for any $K > 1$, 
\[
\abs{H(x)(t)} = 1 < K \qquad \text{for all $t \in \reals$ and all signals $x$}. 
\]
\end{solution}

\item Let $x$ be a signal with period $T$ that is not equal to zero almost everywhere.  Show that $x$ is not absolutely integrable.
\begin{solution}
This is plain and does not really require further explanation, but I've found some students desire more rigour.  

Since $x$ does not equal to zero almost everywhere there exist some finite real numbers $a$ and $b$  such that $\int_{a}^b \abs{x(t)}dt = C > 0$.  Let $k$ be an integer such $-kT < a$ and $kT > b$ so that the integral over $2k+1$ periods 
\[
\int_{-kT}^{kT}\abs{x(t)}dt \geq \int_{a}^b \abs{x(t)}dt = C > 0.
\] 
Now, since $x$ has period $T$
\[
\int_{-ckT}^{ckT}\abs{x(t)}dt = (2c+1) \int_{-kT}^{kT}\abs{x(t)}dt \geq (2c+1)C > 0
\] 
for integers $c$ and since this integral is increasing monotonically with $c$ we have $\int_{-ckT}^{ckT}\abs{x(t)}dt \geq \floor{2c+1}C$ for all $c \in \reals$ where $\floor{2c+1}$ denotes the largest integer less than or equal to $2c+1$.  Now,
\[
\|x\|_1 = \int_{-\infty}^{\infty} \abs{x(t)} dt = \lim_{c \to \infty} \int_{-ckT}^{ckT}\abs{x(t)}dt \geq \lim_{c \to \infty} \floor{2c+1}C = \infty,
\]
and so, $x$ is not absolutely integrable.

%BLERG.  A similar approach can be used to show that x is not square integrable, but you need to use that the function spaces $L^2(T) \subset L^1(T)$ to get an equivalent constant $C$ from assumptions of $x\neq 0 a.e.$.
\end{solution}

\end{excersizelist}

% Start your notes here.
%BLERG: TO DO
% Potentially add to Section 1, definition of limit, continuity, absolute continuity, convergence of signal (pointwise and uniform), and continuity of systems (pointwise, uniform etc).
% We assume/hypothesise that practical systems (electrical, mechanical etc)  are linear and time invariant.  This does not follow from the differential equations.  It's a hypothesis that should be tested like any other.  It's also likely necessary to assume continuity and single valuedness,   See Zemanian Chapter 10 - Passive Systems.  It might be worth writting something on this.


%%% Local Variables: 
%%% mode: latex
%%% TeX-master: "main.tex"
%%% End: 
