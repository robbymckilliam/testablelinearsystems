%\RequirePackage[l2tabu, orthodox]{nag}
\documentclass[11pt,a4paper]{book}
%\documentclass[11pt,a4paper]{memoir}
%\documentclass{amsbook}
%\documentclass{article}
\usepackage{amsmath, amssymb, amsthm, amsfonts}

%\usepackage{fullpage}
%\usepackage{times}
%\usepackage[squaren]{SIunits}
\usepackage[binary-units=true]{siunitx}
\usepackage{circuitikz}
\usepackage{tikz}
\usetikzlibrary{calc,patterns,decorations.pathmorphing,decorations.markings,dsp,chains}
%\usetikzlibrary{external}
%\tikzexternalize % activate!
%\tikzexternalize[prefix=tikzexternal/]
%\tikzset{external/mode=list and make}
%\tikzset{external/check=diff}
%\tikzset{external/force remake}

\usepackage{mdframed}
\usepackage{animate}

%turn animations on/off
\newif\ifanimations %default is off
\animationstrue %uncomment to turn animations on.
\ifanimations
\newcommand{\defaultframerate}{16}
\newcommand{\defaultanimation}[1]{\animategraphics[autoplay,loop]{\defaultframerate}{#1}{}{}}
\else
\newcommand{\defaultanimation}[1]{\includegraphics[page=1]{#1}}
\fi

% Define a the counter cnt. Used to identify files generated for use
% with Gnuplot.
\newcounter{cnt}
\setcounter{cnt}{0}
\makeatletter
\newcommand{\gettikzxy}[3]{%
  \tikz@scan@one@point\pgfutil@firstofone#1\relax
  \edef#2{\the\pgf@x}%
  \edef#3{\the\pgf@y}%
}
\usepackage{pgfplots}
\pgfplotsset{
compat=newest
%every axis/.append style={y tick label style={set thousands separator={}}}
}
%\usepgflibrary{fpu}
\makeatother

\usepackage[small]{caption}
\usepackage{booktabs}
\usepackage{enumitem}
\usepackage{placeins}

\bibliographystyle{robbythesis}
\usepackage[square]{natbib}

\usepackage[hidelinks]{hyperref}

% You can add more of these if it is helpful.
\theoremstyle{plain}
\newtheorem{theorem}{Theorem}
\newtheorem{definition}{Definition}
\newtheorem{lemma}{Lemma}
\newtheorem{corollary}{Corollary}
\newtheorem{result}{Result}
%\newmdtheoremenv[skipabove=\baselineskip, skipbelow=\baselineskip]{test}{Test}
\numberwithin{equation}{section}

%some math functions and symbols
\newcommand{\reals}{{\mathbb R}}
\newcommand{\ints}{{\mathbb Z}}
\newcommand{\naturals}{{\mathbb N}}
\newcommand{\complex}{{\mathbb C}}
\newcommand{\integers}{{\mathbb Z}}
\renewcommand{\Re}{\operatorname{Re}}
\renewcommand{\Im}{\operatorname{Im}}
\newcommand{\atan}[1]{\operatorname{atan}}
\newcommand{\rect}{\Pi}
\newcommand{\sinc}{\operatorname{sinc}}

\newcommand{\term}{\textbf}
\newcommand{\var}{\operatorname{var}}
\newcommand{\covar}{\operatorname{covar}}
%\newcommand{\prob}{{\mathbb P}}
\newcommand{\prob}{\operatorname{Pr}}
\newcommand{\expect}{{\mathbb E}}
\newcommand{\dealias}{\operatorname{dealias}}
\renewcommand{\mid}{\; ; \;}

% Brackets
\newcommand{\br}[1]{{\left( #1 \right)}}
\newcommand{\sqbr}[1]{{\left[ #1 \right]}}
\newcommand{\cubr}[1]{{\left\{ #1 \right\}}}
\newcommand{\scubr}[1]{{\{ #1 \}}}
\newcommand{\abr}[1]{\left< #1 \right>}
\newcommand{\abs}[1]{\left\vert #1 \right\vert}
\newcommand{\sabs}[1]{\vert #1 \vert}
 \newcommand{\floor}[1]{{\left\lfloor #1 \right\rfloor}}
\newcommand{\ceiling}[1]{{\left\lceil #1 \right\rceil}}
\newcommand{\ceil}[1]{\lceil #1 \rceil}
\newcommand{\round}[1]{{\left\lceil #1 \right\rfloor}}
\newcommand{\magn}[1]{\left\| #1 \right\|}
\newcommand{\fracpart}[1]{\left\langle #1 \right\rangle}
\newcommand{\sfracpart}[1]{\langle #1 \rangle}

\newcommand*{\calL}{\mathcal L}
\newcommand{\calD}{{\mathcal D}}
\newcommand{\calF}{{\mathcal F}}
\newcommand{\calZ}{{\mathcal Z}}

%create pgf cross shape
\usetikzlibrary{shapes.misc}

\tikzset{cross/.style={cross out, draw=black, minimum size=2*(#1-\pgflinewidth), inner sep=0pt, outer sep=0pt},
%default radius will be 1pt. 
cross/.default={1pt}
}

%tikz macros
\newcommand{\polezeroaxis}[1]{   
  \draw [<->] (-#1,0) -- (#1,0) node [above left]  {$\Re$};
  \draw [<->] (0,-#1) -- (0,#1) node [below right] {$\Im$};
 }
\newcommand{\zerotikz}[2]{ \node[draw,thick,circle,inner sep=1.75pt] at (#1,#2) {}; }
\newcommand{\poletikz}[2]{ \node[draw,thick,cross=0.1cm] at (#1,#2) {}; }

\newcommand{\vtick}[1]{\draw (#1,-0.075) -- (#1,0.075) }
\newcommand{\htick}[1]{\draw (-0.075,#1) -- (0.075,#1)}

\newcommand{\onebf}{\mathbf{1}}

\newenvironment{solution}{\begin{footnotesize}\textbf{Solution:}}{\end{footnotesize}}

\usepackage{framed,xcolor}
\definecolor{shadecolor}{gray}{0.92}
\newcommand{\todo}[1]{\textcolor{red}{TODO: #1}}
%\usepackage{shaded}

%enviroment for the practical tests (in a grey box)
\newcounter{test}
\newenvironment{test}{
\begin{shaded}\refstepcounter{test}\par\noindent%
\textbf{Test \thetest}
}{
\end{shaded}
}

\title{Testable linear shift invariant systems (Exercise~Solutions)}
\author{Robby McKilliam}

% setup *'d (advanced) sections
% turn advanced section on/off
\newif\ifadvanced %default is off
%\advancedtrue %uncomment to turn advanced sections on.
\usepackage[pagestyles,raggedright]{titlesec}
\usepackage{titletoc}
%\usepackage{titleps}
\newcommand{\secmark}{}
\newcommand{\marktotoc}[1]{\renewcommand{\secmark}{#1}}
\ifadvanced
\newenvironment{advanced}{
  \renewcommand{\secmark}{*}%
  \addtocontents{toc}{\protect\marktotoc{*}}
}
{\addtocontents{toc}{\protect\marktotoc{}}}
\else
\excludecomment{advanced}
\fi
\titleformat{\section}{\normalfont\Large}{\makebox[1.5em][l]{\llap{\secmark}\thesection}}{0.4em}{}
\titlecontents{section}[3.7em]{}{\contentslabel[\llap{\secmark}\thecontentslabel]{2.3em}}{\hspace*{-2.3em}}{\titlerule*{}\bfseries\contentspage}
\newpagestyle{front}{%
  %\headrule
  \sethead[\thepage][][\subsectiontitle]{\sectiontitle}{}{\thepage}
  \setfoot[][][]{}{}{}
}

\newpagestyle{main}{%
  %\headrule
  \sethead[\thepage][][\mytitle]{\llap{\secmark}\thesection\quad\sectiontitle}{}{\thepage}
  \setfoot[][][]{}{}{}
}
\newpagestyle{back}{%
  %\headrule
  \sethead[\thepage][][\mytitle]{\sectiontitle}{}{\thepage}
  \setfoot[][][]{}{}{}
}

%for randomly floating stuff, e.g. the tests 
\newenvironment{randomfloat}{
\begin{figure}
}{
\end{figure}
\addtocounter{figure}{-1}
}

\newenvironment{excersizelist}{%
  \renewcommand*{\theenumi}{\thechapter.\arabic{enumi}}%
  \newcommand\itemadvanced{\stepcounter{enumi}\item[$\ast$\, \theenumi.]}
  %\renewcommand*{\theenumi}{\thechapter.\arabic{enumi}}%
  %\renewcommand*{\labelenumi}{(\arabic{enumi})}%
  \begin{enumerate}
}{%
  \end{enumerate}
}
