\section*{Exercises}

\begin{excersizelist}

% \item As seen in \eqref{eq:utimprespstep} and Figure~\ref{fig:rectpulsedelta} the impulse response of the integrator $I_\infty$ is the step function $u(t)$.  Is the mode of convergence for the limit in~\eqref{eq:utimprespstep} pointwise?  Is it uniform? (see Section~\ref{sec:convergence-signals} for definitions of these modes of convergence).
% \begin{solution}
% The convergence is pointwise.  If $t \leq 0$ then $I_\infty(p_\gamma,t) = 0 = u(t)$ so only remains to consider what happens for $t > 0$.  Fix $t > 0$ and for any $\epsilon > 0$ we can choose $\gamma$ large enough that $I_\infty(p_\gamma,t) = 1$
% \end{solution}

\item \label{excer:distributivecommutive} Show that convolution distributes with addition and commutes with scalar multiplication, that is, show that $a(x*w) + b(y*w) = (ax+by)*w$.
\begin{solution}
\begin{align*}
a (x * w) + b (y * w) &= a \int_{-\infty}^{\infty} x(\tau) w(t - \tau) d\tau + b \int_{-\infty}^{\infty} y(\tau) w(t - \tau) d\tau \\
&= \int_{-\infty}^{\infty} \big(ax(\tau) + by(\tau)\big) w(t - \tau) d\tau\\
&= (ax + by) * w.
\end{align*}
\end{solution}

\item \label{excer:convassociative} Show that convolution is associative.  That is, if $x,y,z$ are signals then $x*(y*z) = (x*y)*z$. 
\begin{solution}
\begin{align*}
(x*y)*z &= \int_{-\infty}^\infty (x*y)(\tau) z(t - \tau) d\tau \\
&= \int_{-\infty}^\infty \int_{-\infty}^\infty x(\kappa) y(\tau-\kappa) z(t - \tau) d\kappa d\tau \\
&= \int_{-\infty}^\infty x(\kappa)  \int_{-\infty}^\infty y(\tau-\kappa) z(t - \tau)  d\tau d\kappa  \qquad \text{(swap order of integration)} \\
&= \int_{-\infty}^\infty x(\kappa)  \int_{-\infty}^\infty y(\nu) z(t - \kappa - \nu)  d\tau d\kappa  \qquad \text{(change variable $\nu = \tau-\kappa$)} \\
&= \int_{-\infty}^\infty x(\kappa)  (y * z)(t - \kappa) d\kappa \\
&= x*(y*z).
\end{align*}
The exchange of integration order can be justified using Fubini's theorem whenever the all of the convolutions involved in $x*(y*z) = (x*y)*z$ exist.
\end{solution}

\item \label{excer:bibostableimpulseresp} Show that a regular system is stable if and only if its impulse response is absolutely integrable.
\begin{solution}
Let $H$ be a regular system and $h$ its impulse response.  If $h$ is absolutely integrable then for all signals $x$ such that $\sabs{x(t)} < M$ for all $t$,
\begin{align*}
H(x,t) &= h * x \\
&= \int_{-\infty}^\infty h(\tau)  x(t - \tau) d\tau \\
&\leq \int_{-\infty}^\infty \sabs{h(\tau) x(t - \tau)} d\tau \\
&\leq \int_{-\infty}^\infty M \sabs{h(\tau)} d\tau \\
&= M \|h\|_1
\end{align*}
for all $t$, and so $H(x,t)$ is bounded.  On the other hand if $h$ is not absolutely integrable then the bounded signal 
\[
s(t) = \begin{cases}
1 & h(-t) > 0 \\
-1 & h(-t) \leq 0
\end{cases}
\]
is such that
\[
H(x,0) = \int_{-\infty}^\infty h(\tau)  s(-\tau) d\tau = \int_{-\infty}^\infty \abs{h(\tau)} d\tau = \infty,
\]
and so the signal $H(x)$ is not bounded at $t = 0$.
\end{solution}

\item Show that the system $H(x) = \int_{-1}^{1} \sin(\pi\tau) x(t + \tau) d\tau$ is linear time invariant and regular.  Find and sketch the impulse response and the step response.
\begin{solution}
The easy way is to spot the impulse response directly.  Observe that
\begin{align*}
H(x)(t) &= \int_{-1}^{1} \sin(\pi\tau) x(t + \tau) d\tau \\
&= \int_{-\infty}^\infty \rect(\tau/2) \sin(\pi\tau) x(t + \tau) d\tau\\
&= -\int_{\infty}^{-\infty} \rect(-\tau/2) \sin(-\pi\tau) x(t - \tau) d\tau \qquad \text{(ch. var. $\tau \to -\tau$)} \\
&= -\int_{-\infty}^{\infty} \rect(\tau/2) \sin(\pi\tau) x(t - \tau) d\tau \\
&= (h * x)(t),
\end{align*}
where we put $h(t) = -\rect(t/2) \sin(\pi t)$.  It follows that $h$ is the impulse response of $H$.  Since $h$ has an impulse resposne it is regular, and since it is regular its also linear and time invariant.

The hard way is to first show linear, then show time invariance, and then find this impulse response using~\eqref{eq:defnimpulseresponse}.  We have
\begin{align*}
H(ax+by) &= \int_{-1}^{1} \sin(\pi\tau) \big( ax(t + \tau) + by(t + \tau) \big) d\tau \\
&= a \int_{-1}^{1} \sin(\pi\tau) x(t + \tau) d\tau + b\int_{-1}^{1} \sin(\pi\tau)  y(t + \tau) \big) d\tau \\
&= aH(x) + bH(y),
\end{align*}
and so, $H$ is linear.  We also have
\begin{align*}
H\big(T_k(x)\big) &= \int_{-1}^{1} \sin(\pi\tau) T_k(x)(t + \tau) d\tau \\
&= \int_{-1}^{1} \sin(\pi\tau) x(t + \tau - k) d\tau \\
&= T_k \left( \int_{-1}^{1} \sin(\pi\tau) x(t + \tau) d\tau   \right) \\
&= T_k\big(H(x)\big),
\end{align*}
and so, $H$ is time invariant.  Now, if $H$ is regular then its impulse response is $h = \lim_{\gamma \rightarrow \infty} H(p_\gamma)$.  Let $h_\gamma$ be the signal
\[
h_\gamma(t) = \int_{-1}^{1} \sin(\pi\tau) p_\gamma(t + \tau) d\tau.
\]
The impulse response exists if $h_\gamma$ converges for each fixed $t$ as $\gamma \to \infty$.  %(BLERG: I'm pretty sure what's needed here is convergence in measure).  
Now, $p_\gamma(t+\tau) = \gamma$ for $t + \tau \in [0, \tfrac{1}{\gamma})$, i.e. $\tau \in [-t, \tfrac{1}{\gamma}-t)$, and zero otherwise.  The integral ranges from $-1$ to $1$ so we are also interested in those $\tau \in [-1,1]$.  When $t > \tfrac{1}{\gamma} +1$ or $t < -1$ the intervals $[-1,1]$ and  $[-t, \tfrac{1}{\gamma}-t)$ are disjoint and we obtain $h(t) = 0$.  Otherwise, when $[-t, \tfrac{1}{\gamma} - t) \subset [-1,1]$, i.e, $-t > -1$ and $\tfrac{1}{\gamma} - t < 1$ we obtain
\begin{align*}
h_\gamma(t) &= \int_{-1}^{1} \sin(\pi\tau) p_\gamma(t + \tau) d\tau \\
&= \gamma \int_{-t}^{1/\gamma - t} \sin(\pi\tau) d\tau \\
&= - \frac{\gamma}{\pi} \big( \cos\big(\pi(1/\gamma - t)\big) - \cos(-\pi t) \big) \\
&= - \frac{\gamma}{\pi} \big( \cos\big(\pi(t - \tfrac{1}{\gamma})\big) - \cos(\pi t) \big).
\end{align*}
Put $\Delta = -\tfrac{1}{\gamma}$ and 
\[
h_\gamma(t) = \frac{1}{\pi} \frac{ \cos\big(\pi(t + \delta)\big) - \cos(\pi t) }{\delta}.
\]
Recognising the limit as $\gamma \to \infty$, or equivalently as $\Delta \to 0$ as
\[
\lim_{\delta \to 0} \frac{ \cos\big(\pi(t + \delta)\big) - \cos(\pi t) }{\delta} = \frac{d}{dt} \cos(\pi t)
\]
we immediately have
\[
\lim_{\gamma \to \infty} h_\gamma(t) = h(t) =\frac{1}{\pi} \frac{d}{dt} \cos(\pi t) = -\sin(\pi t).
\]
on the interval $t \in [\tfrac{1}{\gamma}-1 ,1)$.  It remains to show what happens on the interval $[-1, \tfrac{1}{\gamma}-1)$ that shrinks as $\gamma \to \infty$.

\begin{center}
%\newcommand{\hgamma}[1]{\draw[smooth,color=black,thick,dashed,domain=-1+(1/#1):1,samples=40] plot function{#1/pi*(cos(pi*(x - 1/#1)) - cos(pi*x))};}
\newcommand{\hgamma}[1]{\draw[smooth,color=black,thick,dashed,domain=-1+(1.0/#1):1,samples=40] plot function{-(#1/pi)*(cos(pi*(x-(1.0/#1)))-cos(pi*x)) };}
  \begin{tikzpicture}
    \begin{scope}[yscale=3, xscale=3]
      % \def\sinc(#1){ifthenelse(abs(#1)>0.0001,sin(3.1415926*#1 r)/(3.1415926*#1),1)} %step function
      \draw[->] (-2,0) -- (2,0) node[above] {$t$};
      \draw[->] (0,-0.2) -- (0,1.25);
      \draw[smooth,color=black,thick] (-1.8,0)--(-1,0)--plot[domain=-1:1,samples=40] function{-sin(pi*x)}--(1,0)--(1.8,0);
      \hgamma{2}
      \hgamma{5}
      \hgamma{15}
    \end{scope}
    \htick{1} node[pos=0.5,above right] {$1$};
      \begin{scope}[yscale=2]
      \vtick{1} node[pos=0.5,below right] {$1$};
      \vtick{-1} node[pos=0.5,below right] {$-1$};
    \end{scope}
    
  \end{tikzpicture}
\end{center}

The step response can be found directly by inputing the step function $u$ to the system.  That is
\[ 
H(u,t) = \int_{-1}^{1} \sin(\pi\tau) u(t + \tau) d\tau.
\]  
To find an explicit expression for this integral 3 cases must be considered separately.  Observe that $u(t + \tau)$ is nozero only when $\tau > -t$.  If $t < -1$ then $u(t + \tau) = 0$ for all $\tau \in [-1,1]$ and so
\[
H(u,t) = \int_{-1}^{1} \sin(\pi\tau) u(t + \tau) d\tau = 0 \qquad t < -1.
\] 
If $t > 1$ then $u(t + \tau) = 1$ for all $\tau \in [-1,1]$ and so
\[
H(u,t) = \int_{-1}^{1} \sin(\pi\tau) d\tau = -\frac{\cos(\pi\tau)}{\pi} \big\vert_{-1}^1 = \frac{-\cos(\pi) + \cos(-\pi)}{\pi} = 0 \qquad t > 1.
\]
Finally, if $-1 \leq t \leq 1$ then $u(t + \tau)$ is $1$ for $\tau \in [-t,1]$ and $0$ for $\tau \in [-1,-t)$ and so
\begin{align*}
H(u,t) &= \int_{-t}^{1} \sin(\pi\tau) d\tau \\
&= -\frac{\cos(\pi\tau)}{\pi} \big\vert_{-t}^1 \\
&= \frac{-\cos(\pi) + \cos(-\pi t)}{\pi} = \frac{\cos(\pi t) + 1}{\pi} \qquad 1 \leq t \leq 1.
\end{align*}

An alternative way to find the step response is to apply the integrator system $I_\infty$ to the impulse response $h(t) = -\rect(t/2) \sin(\pi t)$ we derived earlier.  We have
\[
H(u) = I_\infty(h) = -\int_{-\infty}^t \rect(\tau/2) \sin(\pi \tau) d\tau.
\]
Again the integral needs to be split into cases.  When $t < -1$ the $\rect(\tau/2)$ occuring inside the integral is always zero and so $H(u,t) = 0$ for $t < -1$.  When $t > 1$ 
\[
H(u) = -\int_{-1}^1 \sin(\pi \tau) d\tau = 0.
\]
Finally, when $-1 \leq t \leq 1$ we have
\[
H(u) = -\int_{-1}^t \sin(\pi \tau) d\tau  = \frac{\cos(\pi \tau)}{\pi} \big\vert_{-1}^t = \frac{\cos(\pi t) + 1}{\pi}.
\]
Observe that this is the same as previously.  The step response is plotted below.

\begin{center}
  \begin{tikzpicture}
\def\scalex{3}
\def\scaley{5}
    \begin{scope}[yscale=\scaley, xscale=\scalex]
      % \def\sinc(#1){ifthenelse(abs(#1)>0.0001,sin(3.1415926*#1 r)/(3.1415926*#1),1)} %step function
      \draw[->] (-2,0) -- (2,0) node[above] {$t$};
      \draw[->] (0,-0.2) -- (0,2.5/pi);
      \draw[smooth,color=black,thick] (-1.8,0)--(-1,0) -- plot[domain=-1:1,samples=40] function{(cos(pi*x)+1)/pi} -- (1,0) -- (1.8,0);
    \end{scope}
    \begin{scope}[yscale=\scaley]
         \htick{2/pi} node[pos=0.5,above right] {$\tfrac{2}{\pi}$};
       \end{scope}
       \begin{scope}[xscale=\scalex]
         \vtick{1} node[pos=0.5,below right] {$1$};
         \vtick{-1} node[pos=0.5,below right] {$-1$};
       \end{scope}
    
  \end{tikzpicture}
\end{center}


\end{solution}

\item \label{excer:sumgeomeebeta} Show that $\sum_{\ell = 1}^L e^{\beta \ell} = \frac{e^{\beta (L+1)} - e^\beta}{e^\beta - 1}$  (Hint: sum a geometric progression).
\begin{solution}
Put $r = e^\beta$ and put
\[
S_L = \sum_{\ell = 1}^L e^{\beta \ell} = \sum_{\ell = 1}^L r^\ell.
\]
This is the sum of the first $L$ terms of a geometric progression.  We have
\[
r S_L - S_L = r^{L+1} - r
\]
and so
\[
S_L = \frac{r^{L+1} - r}{r - 1} =  \frac{e^{\beta (L+1)} - e^\beta}{e^\beta - 1}
\]
as required.
\end{solution}


\item \label{excer:sumsinegeomean} Show that 
\[
\frac{2j}{L}\sum_{\ell=1}^{L}\sin( \gamma \ell - \theta)  e^{-j \gamma \ell} = \alpha + \alpha^*C
\]
where $\alpha = e^{-j\theta}$ and $C = e^{-j\gamma (L+1)}\frac{\sin(\gamma L)}{L\sin(\gamma)}$. (Hint: solve Exercise~\ref{excer:sumgeomeebeta} first and then use the formula $2j\sin(x) = e^{jx} -  e^{-jx}$).
\begin{solution}
We have 
\[
2j\sin(\gamma \ell - \theta) = e^{j(\gamma \ell - \theta)} -  e^{-j(\gamma \ell - \theta)}
\]
and so the sum becomes
\begin{align*}
\frac{1}{L}\sum_{\ell=1}^{L}(e^{j(\gamma \ell - \theta)} - e^{-j(\gamma \ell - \theta)})  e^{-j \gamma \ell} &= \frac{1}{L}\sum_{\ell=1}^{L}e^{-j\theta} - \frac{1}{L}\sum_{\ell=1}^{L}e^{-2j\gamma}e^{j\theta} \\
&= \alpha - \frac{\alpha^*}{L}\sum_{\ell=1}^{L}e^{-2j\gamma}.
\end{align*}
The sum is a geometric progression and, using the answer to Exercise~\ref{excer:sumgeomeebeta}, we have
\[
\sum_{\ell=1}^{L}e^{-2j\gamma} = \frac{e^{-2j\gamma (L+1)} - e^{-2j\gamma})}{e^{-2j\gamma} - 1}.
\]
The denominator satisfies
\[
e^{-2j\gamma} - 1 = e^{-j\gamma}(e^{-j\gamma} - e^{j\gamma}) = -2 j e^{-j\gamma} \sin(\gamma). 
\]
The numerator satisfies
\begin{align*}
e^{-2j\gamma (L+1)} - e^{-2j\gamma} &= e^{-2j\gamma}(e^{-2j\gamma L} - 1) \\
&= e^{-2j\gamma}e^{-j\gamma L} (e^{-j\gamma L} - e^{j\gamma L}) \\
&= -2 j e^{-j\gamma(L+2)} \sin(\gamma L).
\end{align*}
Thus
\[
\sum_{\ell=1}^{L}e^{-2j\gamma} = \frac{-2 j e^{-j\gamma(L+2)} \sin(\gamma L)}{-2 j e^{-j\gamma} \sin(\gamma)} = \frac{e^{-j\gamma(L+1)} \sin(\gamma L)}{ \sin(\gamma)} = L C
\]
where $C$ is defined in the question statement.  Now
\[
\frac{2j}{L}\sum_{\ell=1}^{L}\sin( \gamma \ell - \theta)  e^{-j \gamma \ell} = \alpha - \frac{\alpha^*}{L}LC = \alpha - \alpha^*C
\]
as required.
\end{solution}

% \item \label{excer:activeRCinverseislinear}  Let $H$ be a system describing the mapping from input voltage signal $x$ to output voltage signal $y = H(x)$ for the active RC electrical circuit described by the differential equation
% \[
% x = -y - R C D(y).
% \] 
% Show that $H$ is linear.
% \begin{solution}
% For two differentiable output signals $y_1$ and $y_2$ put
% \[
% x_1 = -y_1 - R C D(y_1), \qquad x_2 = -y_2 - R C D(y_2),
% \]
% and so, $y_1 = H(x_1)$ and $y_2 = H(x_2)$ by definition of $H$.  Now
% \begin{align*}
% ax_1 + bx_2 &=  -ay_1 - by_2  - R C aD(y_1) - R C bD(y_2) \\
% &= -(ay_1 + by_2)  - R C D(ay_1 + by_2)
% \end{align*}
% and so, $a y_1 + b y_2 = H(a x_1 + b x_2)$ by definition of $H$.  Putting these together
% \[
% a H(x_1) + b H(x_2) = H(a x_1 + b x_2),
% \]
% and so, $H$ is linear.
% \end{solution}

% \item \label{excer:explicityasolution} Find an explicit formula for the imaginary part of the signal $y_a$ from~\eqref{eq:complexenvelopeya}.
% \begin{solution}
% Define magnitude and phases $A_1, \phi_1$ and $A_2 , \phi_2$ such that
% \[
% A_1e^{j\phi_1} = -\frac{1}{3 + 6\pi R C f_1 j}, \qquad A_2e^{j\phi_2} = -\frac{1}{3 + 6\pi R C f_2 j},
% \]
% that is
% \[
% A_1 = \frac{1}{3}\big(1 + 4\pi^2 R^2 C^2 f_1^2\big)^{-\tfrac{1}{2}}, \qquad \phi_1 = \arctan\big( 2\pi R C f_1 \big) + \pi,
% \]
% \[
% A_2 = \frac{1}{3}\big(1 + 4\pi^2 R^2 C^2 f_2^2\big)^{-\tfrac{1}{2}}, \qquad \phi_2 = \arctan\big( 2\pi R C f_2 \big) + \pi.
% \]
% Then
% \[
% y_a(t) = A_1 e^{(2\pi f_1 t + \phi_1)j} + A_2 e^{(2\pi f_2 t + \phi_2)j}.
% \]
% Now
% \[
% y(t) = \Im(y_a(t)) = A_1 \sin(2\pi f_1 t + \phi_1) + A_2 \sin(2\pi f_2 t + \phi_2).
% \]
% \end{solution}


% \item
% \\ \begin{solution} To see why the definition above makes sense, let $H$ be a regular system with impulse response $h$.  The response of $H$ to $p_\gamma$ is
% \begin{align*}
% H(p_\gamma,t) &= \int_{-\infty}^{\infty} p_\gamma(\tau) h(t - \tau) d\tau \\
% &= \gamma \int_{0}^{1/\gamma} h(t - \tau) d\tau
% \rightarrow h(t) \qquad \text{as $\gamma \rigtharrow \infty$}.
% \end{align*}
% and, since signals are peicewise continuous with limits from the left, for any $\epsilon$ we can choose $\gamma$ large enough that $\abs{ h(t - \tau) - h(t)} < \epsilon$ for all $\tau \in [0,\frac{1}{\gamma}]$ and so
% \begin{align*}
% H(p_\gamma,t) &= \gamma \int_{0}^{1/\gamma} h(t - \tau) - h(t) + h(t) d\tau \\
%  &= \gamma \int_{0}^{1/\gamma} h(t) d\tau + \gamma  \int_{0}^{1/\gamma} h(t - \tau) - h(t) d\tau\\
% \end{align*}
% and since 
% \[
% \abs{ \int_{0}^{1/\gamma} h(t - \tau) - h(t) d\tau } \leq \int_{0}^{1/\gamma} \sabs{h(t - \tau) - h(t)} d\tau < \epsilon
% \]
% and $\epilon$ can be chosen arbitrailty small
% \end{solution}

\end{excersizelist}

%%% Local Variables: 
%%% mode: latex
%%% TeX-master: "main.tex"
%%% End: 

